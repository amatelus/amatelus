\chapter{did:amt Method Specification}

\section{Abstract}

\begin{definition}
  \label{def:did-amt}
  The \texttt{did:amt} method is a Decentralized Identifier (DID) that is algorithmically generated
  and resolved without reliance on any external Verifiable Data Registry (VDR) such as a blockchain.
  This method is designed for high-stakes environments (public administration, government) where
  data integrity and operational robustness are paramount.
  \uses{def:amatelus}
  \lean{AMATELUS.DID}
  \leanok
\end{definition}

\section{did:amt Syntax}

The \texttt{did:amt} syntax conforms to the W3C DID Core specification:

\begin{verbatim}
did-amt              := "did:amt:" method-specific-id
method-specific-id   := crockford-base32-encoded-sha3-512-hash
\end{verbatim}

The \texttt{method-specific-id} is a Crockford's Base32 encoded string of the hash value generated through the local creation process.

\subsection{Crockford's Base32 Character Set}

\texttt{0123456789ABCDEFGHJKMNPQRSTVWXYZ}

This character set is chosen to minimize human transcription errors in administrative settings (e.g., avoiding confusion between \texttt{O} and \texttt{0}, or \texttt{I} and \texttt{l}).

\section{CRUD Operations}

\subsection{Create: Local Generation}

\begin{theorem}
  \label{thm:did-local-generation}
  A \texttt{did:amt} identifier is generated locally on the owner's device without network registration.

  The generation process consists of:
  \begin{enumerate}
    \item Generate Ed25519 key pair
    \item Prepare information pair: AMT Version Number + Public Key
    \item Select DID Document template corresponding to AMT version
    \item Derive DID through local cryptographic operations
  \end{enumerate}
  \uses{def:did-amt, def:crypto-assumptions}
  \lean{AMATELUS.DID}
  \leanok
\end{theorem}

\subsection{Read: Local Resolution}

\begin{theorem}
  \label{thm:did-local-resolution}
  The resolution of a \texttt{did:amt} is completed locally by a verifier.

  No external Verifiable Data Registry (blockchain, centralized service) or network calls are required.
  Verifier receives the \texttt{[AMT Version Number, Public Key]} pair from the owner and executes
  the same local derivation steps for verification.
  \uses{def:did-amt, def:crypto-assumptions}
  \lean{AMATELUS.DID}
  \leanok
\end{theorem}

\subsection{Update: Not Supported}

As \texttt{did:amt} DID Documents are immutable, Update operations are not supported. Key rotation is handled by issuing a new DID and linking it via a ``DID Continuity Verifiable Credential'' issued by a trusted third party.

\subsection{Deactivate: Key Destruction}

There is no explicit Deactivate operation. Deactivation is effectively achieved by destroying the associated private key.

\section{Cryptographic Properties}

\subsection{DID Identifier Security}

\begin{itemize}
  \item \textbf{Hash function}: SHA3-512 (post-quantum collision resistance)
  \item \textbf{Security level}: 256-bit (quantum-resistant)
  \item \textbf{Uniqueness}: Permanent uniqueness guaranteed by collision resistance
\end{itemize}

\subsection{DID Ownership Proof}

\begin{itemize}
  \item \textbf{Signature algorithm}: Ed25519 (current version)
  \item \textbf{Security level}: 128-bit (classical only, see future evolution)
  \item \textbf{Signature verification}: Public key validation against DID Document
\end{itemize}

\subsection{Privacy}

The avoidance of a Verifiable Data Registry (VDR) ensures that:
\begin{itemize}
  \item DIDs are not publicly enumerable
  \item No central authority records DID creation
  \item High degree of privacy maintained
\end{itemize}

\subsection{Operational Robustness}

Crockford's Base32 encoding minimizes human transcription errors during manual entry in administrative processes.

\section{Future Evolution: PQC Transition}

\subsection{Versioning for Cryptographic Agility}

The AMT protocol is designed with ``cryptographic agility,'' allowing for the upgrade of its cryptographic suite through versioning.

\subsection{Foreseeable Changes (AMT v1+)}

\subsubsection{Post-Quantum Signature Migration}

The most critical change will be migration from Ed25519 to a NIST-selected PQC signature algorithm (e.g., CRYSTALS-Dilithium). This ensures DID ownership proof is also secure against quantum computers.

\subsubsection{Binary Data Format Challenge}

PQC signature algorithms require significantly larger public key and signature sizes (several to tens of kilobytes). Future versions will likely specify a binary representation format such as CBOR (Concise Binary Object Representation) for DID Documents to maintain efficiency.

\subsubsection{Interoperability Through Versioning}

The \texttt{AMT Version Number} presented by the owner allows verifiers to accurately determine:
\begin{itemize}
  \item Which cryptographic algorithms to use (Ed25519 vs PQC)
  \item Which data formats to expect (JSON-LD vs CBOR)
  \item Secure interoperability during transition periods
\end{itemize}

\subsection{Example: Version 0 DID Document}

\begin{verbatim}
{
  "@context": ["https://www.w3.org/ns/did/v1"],
  "id": "did:amt:0V3R4T7K1Q2P3N4M5J6H7G8F5D4C3B2A...",
  "verificationMethod": [{
    "id": "did:amt:0V3R4T7K1Q2P3N4M5J6H7G8F5D4C3B2A...#key-1",
    "type": "Ed25519VerificationKey2020",
    "controller": "did:amt:0V3R4T7K1Q2P3N4M5J6H7G8F5D4C3B2A...",
    "publicKeyMultibase": "k3t635r7r1c0kdf41n2p5h3t2d3n2g5r..."
  }],
  "authentication": ["#key-1"],
  "assertionMethod": ["#key-1"]
}
\end{verbatim}

