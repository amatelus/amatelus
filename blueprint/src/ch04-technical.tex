\chapter{Technical Specifications}

\section{DID Document Design Principles}

The AMT protocol employs did:amt for distributed identifier generation and resolution. The DID Document structure is minimal and stateless:

\begin{verbatim}
DID := did:amt:H(DIDDoc)

DIDDoc := {
  id: DID,
  publicKey: PublicKey,
  metadata: Metadata
}
\end{verbatim}

where $H$ is SHA3-512 (collision-resistant hash function).

\subsection{Critical Design Decision: Abolition of serviceEndpoint}

Traditional DID specifications include \texttt{serviceEndpoint} in DIDDocuments. The AMT protocol rejects this for principled reasons:

\subsubsection{Problem with serviceEndpoint}

\begin{itemize}
  \item \textbf{Short lifetime}: Service endpoint contracts last months to few years
  \item \textbf{Individual burden}: Users must manage contract renewals and provider changes
  \item \textbf{Unrealistic assumption}: Assumes individual can maintain stable, contracted endpoints
  \item \textbf{Real-world failure}: When provider fails or service terminates, users cannot migrate
\end{itemize}

\subsubsection{AMT Solution}

Remove \texttt{serviceEndpoint} from DIDDocument entirely.

Instead:
\begin{itemize}
  \item DID lifetime: Long-term (public-key-based, years to decades)
  \item Endpoint lifetime: Session-scoped (minutes to hours)
  \item Endpoint management: Service provider responsibility only
  \item Individual burden: Zero
\end{itemize}

\subsubsection{Implementation Reality}

\begin{itemize}
  \item Wallet is invoked by service app (already knows its own endpoint)
  \item DIDDocument contains only public key for cryptographic verification
  \item Endpoint information implicit in application context
  \item No distributed endpoint discovery needed
\end{itemize}

For complete did:amt specification details, see Chapter 2 (did:amt Method Specification).

\section{DIDComm Integration}

AMATELUS employs DIDComm Messaging v2.1 as the sole communication protocol.

\subsection{Message Structure}

\begin{verbatim}
DIDCommMessageSend := {
  senderDID: ValidDID,
  senderDoc: Option<ValidDIDDocument>,
  vcs: List<ValidVC>,
  zkp: Option<ValidZKP>
}
\end{verbatim}

\subsection{Security Properties}

\begin{itemize}
  \item \textbf{ECDH-1PU authenticated encryption}: Sender identity authenticated exclusively to recipient
  \item \textbf{Message-level security}: Independent of transport layer
  \item \textbf{Transport agnosticism}: Works over HTTPS, BLE, WebSocket, etc.
  \item \textbf{Sender anonymity (optional)}: Anoncrypt layer available if needed
\end{itemize}

\section{Verifiable Credentials}

\subsection{VC Structure}

\begin{verbatim}
VC := {
  issuer: DID_issuer,
  subject: DID_subject,
  claims: Claims,
  signature: Signature,
  credentialStatus: RevocationInfo,
  deLinkageInfo: Option<DeLinkageInfo>
}
\end{verbatim}

\subsection{Delinkage Information}

To prevent cross-service correlation:

\begin{verbatim}
DeLinkageInfo := {
  identityDID: DID,        // Long-term identity DID
  communicationDID: DID    // Session-specific DID
}
\end{verbatim}

ZKP proves ownership of both DIDs without revealing either to external parties.

\section{Zero-Knowledge Proofs}

\subsection{Dual Nonce Binding}

\begin{verbatim}
ZKP := (pi, x, nonce_holder, nonce_verifier)
where pi proves knowledge of w such that
R(x, w, nonce_holder, nonce_verifier) = 1
\end{verbatim}

\subsection{Nonce Generation and Binding}

Both Holder and Verifier generate independent nonces ($\geq$128 bits):

\begin{enumerate}
  \item Verifier generates \texttt{nonce2} (challenge)
  \item Verifier sends \texttt{nonce2} to Holder
  \item Holder generates \texttt{nonce1} independently
  \item Holder combines: \texttt{nonce\_combined = H(nonce1 || nonce2)}
  \item ZKP bound to \texttt{nonce\_combined}
  \item Holder returns (ZKP, nonce1) to Verifier
  \item Verifier verifies both nonces cryptographically
\end{enumerate}

\subsection{Security Property: Mutual Defense}

If either party generates unique nonces:
\begin{itemize}
  \item Holder bug (fixed nonce1) $\Rightarrow$ Verifier's nonce2 provides protection
  \item Verifier bug (fixed nonce2) $\Rightarrow$ Holder's nonce1 provides protection
  \item Both unique $\Rightarrow$ Complete protection from replay
\end{itemize}

Neither party depends solely on the other's implementation quality.

\subsection{Computational Efficiency}

\begin{itemize}
  \item \textbf{Offline precomputation}: Heavy circuit evaluation (minutes to hours)
  \item \textbf{Real-time nonce binding}: Light operations only (milliseconds to hundreds of milliseconds)
  \item \textbf{UX compatibility}: User interaction completes within tolerance (3 seconds)
\end{itemize}

