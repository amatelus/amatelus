\chapter{Cryptographic Foundations}

\section{Security Assumptions}

\begin{definition}
  \label{def:crypto-assumptions}
  AMATELUS relies on the following cryptographic security assumptions:
  \begin{itemize}
    \item \textbf{Collision-resistant hash}: SHA3-512 provides 128-bit security against quantum adversaries
    \item \textbf{Unforgeable signatures}: Dilithium2 provides 128-bit security against quantum adversaries
    \item \textbf{ZKP soundness}: Standard zero-knowledge properties (completeness, soundness, zero-knowledge)
  \end{itemize}
  \uses{def:amatelus}
  \lean{AMATELUS.Cryptographic}
  \leanok
\end{definition}

\section{Threat Model and Mitigations}

\subsection{Impersonation Attack (Different Secret Key)}

\begin{theorem}
  \label{thm:impersonation-resistance}
  Impersonation attacks with different secret keys are cryptographically prevented.

  If an attacker uses a different secret key to forge a ZKP, the signature verification will fail
  because DIDComm makes the sender's public key known to the recipient.
  \uses{def:crypto-assumptions}
  \lean{AMATELUS.Cryptographic}
  \leanok
\end{theorem}

\subsection{Replay Attack (Same ZKP, Same User)}

\begin{proposition}
  \label{prop:replay-application-layer}
  Replay attack prevention (preventing legitimate users from reusing the same ZKP) is NOT
  a responsibility of AMATELUS protocol. This is an application-layer responsibility.

  \textbf{Important distinction}:
  \begin{itemize}
    \item \textbf{Impersonation attacks} (attacker with different secret key):
      PREVENTED by AMATELUS through DIDComm
    \item \textbf{Replay attacks} (legitimate user reusing same ZKP):
      NOT handled by AMATELUS. Applications requiring single-use semantics must implement nonce mechanisms.
  \end{itemize}

  Applications that require ZKP single-use guarantees should implement nonce handling at the application layer:
  \begin{itemize}
    \item Generate unique session nonces for each verification
    \item Record and verify nonce freshness (checking that the nonce has not been used before)
    \item Maintain nonce history in application database
  \end{itemize}

  Applications where ZKP reuse is acceptable (e.g., age verification, permission checks, one-time access grants)
  do not require additional replay prevention mechanisms.
  \uses{def:crypto-assumptions}
  \lean{AMATELUS.Cryptographic}
\end{proposition}

\subsection{Man-in-the-Middle Attack}

\begin{proposition}
  \label{prop:mitm-defense}
  Man-in-the-Middle attacks are mitigated at the transport layer.

  While AMATELUS provides cryptographic identity verification, ECDH-1PU authenticated
  encryption and TLS/HTTPS are the responsibility of service providers.
  \uses{def:crypto-assumptions}
  \lean{AMATELUS.Cryptographic}
\end{proposition}

\subsection{Sybil Attack (Multiple DIDs)}

\begin{proposition}
  \label{prop:sybil-resilience}
  Multiple DID possession is intentional protocol design for privacy protection.

  While a single entity can control multiple DIDs, Anonymous Hash Identifiers (AHI)
  restrict per-audit-domain abuse through cryptographic binding to national identity systems.
  \uses{def:amatelus}
  \lean{AMATELUS.Protocol}
\end{proposition}

