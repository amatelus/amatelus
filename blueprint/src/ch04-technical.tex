\chapter{Technical Specifications}


\begin{definition}
  \label{def:tech-chapter}
  This chapter covers technical specifications aspects of AMATELUS.
  \uses{def:did-amt,def:amatelus}
  \lean{AMATELUS.Protocol}
  \leanok
\end{definition}
\section{DID Document Design Principles}

The AMT protocol employs did:amt for distributed identifier generation and resolution. The DID Document structure is minimal and stateless:

\begin{verbatim}
DID := did:amt:H(DIDDoc)

DIDDoc := {
  id: DID,
  publicKey: PublicKey,
  metadata: Metadata
}
\end{verbatim}

where $H$ is SHA3-512 (collision-resistant hash function).

\subsection{Critical Design Decision: Abolition of serviceEndpoint}

Traditional DID specifications include \texttt{serviceEndpoint} in DIDDocuments. The AMT protocol rejects this for principled reasons:

\subsubsection{Problem with serviceEndpoint}

\begin{itemize}
  \item \textbf{Short lifetime}: Service endpoint contracts last months to few years
  \item \textbf{Individual burden}: Users must manage contract renewals and provider changes
  \item \textbf{Unrealistic assumption}: Assumes individual can maintain stable, contracted endpoints
  \item \textbf{Real-world failure}: When provider fails or service terminates, users cannot migrate
\end{itemize}

\subsubsection{AMT Solution}

Remove \texttt{serviceEndpoint} from DIDDocument entirely.

Instead:
\begin{itemize}
  \item DID lifetime: Long-term (public-key-based, years to decades)
  \item Endpoint lifetime: Session-scoped (minutes to hours)
  \item Endpoint management: Service provider responsibility only
  \item Individual burden: Zero
\end{itemize}

\subsubsection{Implementation Reality}

\begin{itemize}
  \item Wallet is invoked by service app (already knows its own endpoint)
  \item DIDDocument contains only public key for cryptographic verification
  \item Endpoint information implicit in application context
  \item No distributed endpoint discovery needed
\end{itemize}

For complete did:amt specification details, see Chapter 2 (did:amt Method Specification).

\section{DIDComm Integration}

AMATELUS employs DIDComm Messaging v2.1 as the sole communication protocol.

\subsection{Message Structure}

\begin{verbatim}
DIDCommMessageSend := {
  senderDID: ValidDID,
  senderDoc: Option<ValidDIDDocument>,
  vcs: List<ValidVC>,
  zkp: Option<ValidZKP>
}
\end{verbatim}

\subsection{Security Properties}

\begin{itemize}
  \item \textbf{ECDH-1PU authenticated encryption}: Sender identity authenticated exclusively to recipient
  \item \textbf{Message-level security}: Independent of transport layer
  \item \textbf{Transport agnosticism}: Works over HTTPS, BLE, WebSocket, etc.
  \item \textbf{Sender anonymity (optional)}: Anoncrypt layer available if needed
\end{itemize}

\section{Verifiable Credentials}

\subsection{VC Structure}

\begin{verbatim}
VC := {
  issuer: DID_issuer,
  subject: DID_subject,
  claims: Claims,
  signature: Signature,
  credentialStatus: RevocationInfo,
  deLinkageInfo: Option<DeLinkageInfo>
}
\end{verbatim}

\subsection{Delinkage Information}

To prevent cross-service correlation:

\begin{verbatim}
DeLinkageInfo := {
  identityDID: DID,        // Long-term identity DID
  communicationDID: DID    // Session-specific DID
}
\end{verbatim}

ZKP proves ownership of both DIDs without revealing either to external parties.

\section{Zero-Knowledge Proofs}

\subsection{Security Guarantees}

AMATELUS provides cryptographic proof generation and verification. Impersonation attacks are prevented
through DIDComm's requirement for explicit public key transmission.

\begin{theorem}
  \label{thm:didcomm-impersonation-prevention}
  Impersonation attacks (attacker with different secret key) are cryptographically prevented through DIDComm.

  \textbf{Mechanism:}
  \begin{itemize}
    \item Verifier receives ZKP along with sender's public key via DIDComm (\texttt{senderDoc})
    \item ZKP must be cryptographically signed with secret key corresponding to this public key
    \item If attacker uses different secret key, the signature verification fails
    \item Attacker cannot reuse legitimate ZKP with different SK (cryptographically impossible)
  \end{itemize}

  \textbf{Security guarantee}: Impersonation attacks are prevented by DIDComm alone.
  \lean{AMATELUS.Cryptographic}
  \leanok
\end{theorem}

\begin{theorem}
  \label{thm:secret-key-correspondence-certainty}
  DIDComm establishes cryptographic certainty between ZKP and secret key correspondence.

  By requiring explicit transmission of DIDDocument (containing public key), the protocol ensures:
  \begin{itemize}
    \item Verifier definitively knows which public key corresponds to the ZKP
    \item ZKP is bound to exactly one secret key (the corresponding private key)
    \item Any ZKP signed with different secret key will fail verification
  \end{itemize}

  \textbf{Result}: Cryptographic correspondence is certain, making impersonation impossible.
  \lean{AMATELUS.Cryptographic}
  \leanok
\end{theorem}

\subsection{Replay Prevention (Application Layer Responsibility)}

\textbf{Important}: Replay prevention (preventing legitimate users from reusing the same ZKP across multiple sessions)
is NOT part of AMATELUS protocol. This is an application-layer responsibility.

Services that require single-use ZKP semantics (e.g., one-time registrations, authorization grants) should:

\begin{itemize}
  \item Implement nonce-based mechanisms in their application layer
  \item Generate unique session nonces for each verification request
  \item Record and verify nonce freshness (checking that the nonce has not been used before)
  \item Maintain nonce history within their application database
\end{itemize}

Services where ZKP reuse is acceptable (e.g., age verification for each transaction) do not require
nonce mechanisms.

\textbf{Important}: AMATELUS does not specify or enforce nonce handling. This is entirely
the responsibility of the application implementing the protocol.

\subsection{Computational Efficiency}

\begin{itemize}
  \item \textbf{Offline precomputation}: Heavy circuit evaluation (minutes to hours)
  \item \textbf{Real-time verification}: Light operations only (milliseconds to hundreds of milliseconds)
  \item \textbf{UX compatibility}: User interaction completes within tolerance (3 seconds)
\end{itemize}

