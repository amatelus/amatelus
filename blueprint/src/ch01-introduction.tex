\chapter{Introduction}

\section{AMATELUS Protocol Overview}

The AMATELUS protocol is a cryptographic authentication mechanism integrating Decentralized Identifiers (DIDs), Verifiable Credentials (VCs), and Zero-Knowledge Proofs (ZKPs).

\subsection{What AMATELUS Provides}

AMATELUS delivers the following capabilities:
\begin{itemize}
  \item Public key infrastructure (PKI) based distributed identifiers
  \item Zero-knowledge proof-based attribute ownership verification
  \item Verifiable credentials for claim issuance and storage
  \item Cryptographic trust guarantees
\end{itemize}

\subsection{What AMATELUS Does NOT Provide}

Critically, AMATELUS \emph{does not} provide:
\begin{itemize}
  \item Centralized directory services (DID resolution is local-only)
  \item Mediation or relay infrastructure
  \item User authorization decisions
  \item Communication endpoint management
  \item Message delivery guarantees
\end{itemize}

\subsection{Design Philosophy}

AMATELUS is a \emph{cryptographic authentication mechanism}, not a communication infrastructure.

The core design principle emerges from the observation that:
\begin{itemize}
  \item Every real-world service (banking, government, voting, SNS) requires a centralized service provider
  \item AMATELUS provides only cryptographic trust verification
  \item Service providers maintain responsibility for endpoints, authorization, and communication security
  \item Individual users bear zero burden for endpoint contract management
\end{itemize}

This boundary-clarity eliminates the complexity of distributed infrastructure while preserving cryptographic security properties.

