% AMATELUS Protocol Blueprint
% English specification based on formal verification and architectural analysis

\chapter{Nomenclature and Scope}

\section{Terminology}

This document uses two complementary names for the same underlying technology:

\subsection{AMT Protocol (Technical Name)}

\textbf{AMT} (Autonomous Meta-Trust) is the \emph{technical name} for the protocol, designed for international standardization. This is the name used in technical discussions, standards bodies, and peer-reviewed literature.

\subsection{AMATELUS (Marketing Name)}

\textbf{AMATELUS} is the \emph{marketing name}, indicating "Japanese origin." This is used for brand identity, commercial deployment, and public communication in Japan.

\subsection{Relationship}

\begin{itemize}
  \item \textbf{AMT Protocol} = international-standard-oriented technical specification
  \item \textbf{AMATELUS} = Japan-specific implementation and deployment brand
  \item Both refer to the same underlying cryptographic authentication mechanism
\end{itemize}

\section{Document Scope}

This blueprint covers:
\begin{itemize}
  \item The AMT Protocol (technical foundation)
  \item The did:amt DID Method Specification (standardized identifier format)
  \item AMATELUS implementation guidance (Japan-specific deployment)
  \item Cryptographic foundations and formal proofs
\end{itemize}

\chapter{Introduction}

\section{AMATELUS Protocol Overview}

The AMATELUS protocol is a cryptographic authentication mechanism integrating Decentralized Identifiers (DIDs), Verifiable Credentials (VCs), and Zero-Knowledge Proofs (ZKPs).

\subsection{What AMATELUS Provides}

AMATELUS delivers the following capabilities:
\begin{itemize}
  \item Public key infrastructure (PKI) based distributed identifiers
  \item Zero-knowledge proof-based attribute ownership verification
  \item Verifiable credentials for claim issuance and storage
  \item Cryptographic trust guarantees
\end{itemize}

\subsection{What AMATELUS Does NOT Provide}

Critically, AMATELUS \emph{does not} provide:
\begin{itemize}
  \item Centralized directory services (DID resolution is local-only)
  \item Mediation or relay infrastructure
  \item User authorization decisions
  \item Communication endpoint management
  \item Message delivery guarantees
\end{itemize}

\subsection{Design Philosophy}

AMATELUS is a \emph{cryptographic authentication mechanism}, not a communication infrastructure.

The core design principle emerges from the observation that:
\begin{itemize}
  \item Every real-world service (banking, government, voting, SNS) requires a centralized service provider
  \item AMATELUS provides only cryptographic trust verification
  \item Service providers maintain responsibility for endpoints, authorization, and communication security
  \item Individual users bear zero burden for endpoint contract management
\end{itemize}

This boundary-clarity eliminates the complexity of distributed infrastructure while preserving cryptographic security properties.

\chapter{did:amt Method Specification}

\section{Abstract}

The \texttt{did:amt} method specifies a Decentralized Identifier (DID) that is algorithmically generated and resolved without reliance on any external Verifiable Data Registry (VDR) such as a blockchain. This method is based on the AMT (Autonomous Meta-Trust) Protocol, designed for high-stakes environments like public administration, where data integrity and operational robustness are paramount.

\section{did:amt Syntax}

The \texttt{did:amt} syntax conforms to the W3C DID Core specification:

\begin{verbatim}
did-amt              := "did:amt:" method-specific-id
method-specific-id   := crockford-base32-encoded-sha3-512-hash
\end{verbatim}

The \texttt{method-specific-id} is a Crockford's Base32 encoded string of the hash value generated through the local creation process.

\subsection{Crockford's Base32 Character Set}

\texttt{0123456789ABCDEFGHJKMNPQRSTVWXYZ}

This character set is chosen to minimize human transcription errors in administrative settings (e.g., avoiding confusion between \texttt{O} and \texttt{0}, or \texttt{I} and \texttt{l}).

\section{CRUD Operations}

\subsection{Create: Local Generation}

A \texttt{did:amt} identifier is generated locally on the owner's device. No network registration is required.

\begin{enumerate}
  \item Generate Ed25519 key pair
  \item Prepare information pair: AMT Version Number + Public Key
  \item Select DID Document template corresponding to AMT version
  \item Derive DID:
    \begin{enumerate}
      \item Encode Ed25519 public key in Crockford's Base32 with multibase prefix \texttt{k}
      \item Insert into DID Document template
      \item Normalize template using JCS (RFC 8785)
      \item Hash with SHA3-512
      \item Encode hash digest with Crockford's Base32
      \item Prepend \texttt{did:amt:} to obtain final DID
    \end{enumerate}
  \item Finalize DID Document with derived DID
\end{enumerate}

\subsection{Read: Local Resolution}

The resolution of a \texttt{did:amt} is completed locally by a verifier who receives the \texttt{[AMT Version Number, Public Key]} pair from the owner and executes the same derivation steps.

No external service calls required.

\subsection{Update: Not Supported}

As \texttt{did:amt} DID Documents are immutable, Update operations are not supported. Key rotation is handled by issuing a new DID and linking it via a ``DID Continuity Verifiable Credential'' issued by a trusted third party.

\subsection{Deactivate: Key Destruction}

There is no explicit Deactivate operation. Deactivation is effectively achieved by destroying the associated private key.

\section{Cryptographic Properties}

\subsection{DID Identifier Security}

\begin{itemize}
  \item \textbf{Hash function}: SHA3-512 (post-quantum collision resistance)
  \item \textbf{Security level}: 256-bit (quantum-resistant)
  \item \textbf{Uniqueness}: Permanent uniqueness guaranteed by collision resistance
\end{itemize}

\subsection{DID Ownership Proof}

\begin{itemize}
  \item \textbf{Signature algorithm}: Ed25519 (current version)
  \item \textbf{Security level}: 128-bit (classical only, see future evolution)
  \item \textbf{Signature verification}: Public key validation against DID Document
\end{itemize}

\subsection{Privacy}

The avoidance of a Verifiable Data Registry (VDR) ensures that:
\begin{itemize}
  \item DIDs are not publicly enumerable
  \item No central authority records DID creation
  \item High degree of privacy maintained
\end{itemize}

\subsection{Operational Robustness}

Crockford's Base32 encoding minimizes human transcription errors during manual entry in administrative processes.

\section{Future Evolution: PQC Transition}

\subsection{Versioning for Cryptographic Agility}

The AMT protocol is designed with ``cryptographic agility,'' allowing for the upgrade of its cryptographic suite through versioning.

\subsection{Foreseeable Changes (AMT v1+)}

\subsubsection{Post-Quantum Signature Migration}

The most critical change will be migration from Ed25519 to a NIST-selected PQC signature algorithm (e.g., CRYSTALS-Dilithium). This ensures DID ownership proof is also secure against quantum computers.

\subsubsection{Binary Data Format Challenge}

PQC signature algorithms require significantly larger public key and signature sizes (several to tens of kilobytes). Future versions will likely specify a binary representation format such as CBOR (Concise Binary Object Representation) for DID Documents to maintain efficiency.

\subsubsection{Interoperability Through Versioning}

The \texttt{AMT Version Number} presented by the owner allows verifiers to accurately determine:
\begin{itemize}
  \item Which cryptographic algorithms to use (Ed25519 vs PQC)
  \item Which data formats to expect (JSON-LD vs CBOR)
  \item Secure interoperability during transition periods
\end{itemize}

\subsection{Example: Version 0 DID Document}

\begin{verbatim}
{
  "@context": ["https://www.w3.org/ns/did/v1"],
  "id": "did:amt:0V3R4T7K1Q2P3N4M5J6H7G8F5D4C3B2A...",
  "verificationMethod": [{
    "id": "did:amt:0V3R4T7K1Q2P3N4M5J6H7G8F5D4C3B2A...#key-1",
    "type": "Ed25519VerificationKey2020",
    "controller": "did:amt:0V3R4T7K1Q2P3N4M5J6H7G8F5D4C3B2A...",
    "publicKeyMultibase": "k3t635r7r1c0kdf41n2p5h3t2d3n2g5r..."
  }],
  "authentication": ["#key-1"],
  "assertionMethod": ["#key-1"]
}
\end{verbatim}

\chapter{Deployment Models}

\section{Service-Driven Model (Primary)}

In the predominant deployment model:
\begin{enumerate}
  \item Service provider application (bank, government, SNS) initiates
  \item Wallet is invoked from service app via OS-level Intent/deeplink
  \item Wallet generates DIDComm message containing:
    \begin{itemize}
      \item Communication DID (ephemeral or persistent based on service)
      \item DID Document (public key only)
      \item VC/ZKP proving required attributes
    \end{itemize}
  \item DIDComm message sent to service provider (HTTPS, typically)
  \item Service provider performs:
    \begin{itemize}
      \item Cryptographic verification of ZKP/VC
      \item Authorization decision
      \item Service execution
    \end{itemize}
\end{enumerate}

Trust origin: Service provider (centralized, established).

Endpoint lifecycle: Service-scoped (session-based).

\section{Physical Proximity Model (Supplementary)}

For in-person scenarios:
\begin{enumerate}
  \item Bluetooth Low Energy (BLE) provides discovery via physical proximity
  \item Public services (municipal offices, event gates)
  \item Session completes within local network
  \item No persistent endpoint contracts
\end{enumerate}

Examples:
\begin{itemize}
  \item Age verification at municipal counter
  \item Facility access control
  \item In-person credential verification
\end{itemize}

\chapter{Technical Specifications}

\section{DID Document Design Principles}

The AMT protocol employs did:amt for distributed identifier generation and resolution. The DID Document structure is minimal and stateless:

\begin{verbatim}
DID := did:amt:H(DIDDoc)

DIDDoc := {
  id: DID,
  publicKey: PublicKey,
  metadata: Metadata
}
\end{verbatim}

where $H$ is SHA3-512 (collision-resistant hash function).

\subsection{Critical Design Decision: Abolition of serviceEndpoint}

Traditional DID specifications include \texttt{serviceEndpoint} in DIDDocuments. The AMT protocol rejects this for principled reasons:

\subsubsection{Problem with serviceEndpoint}

\begin{itemize}
  \item \textbf{Short lifetime}: Service endpoint contracts last months to few years
  \item \textbf{Individual burden}: Users must manage contract renewals and provider changes
  \item \textbf{Unrealistic assumption}: Assumes individual can maintain stable, contracted endpoints
  \item \textbf{Real-world failure}: When provider fails or service terminates, users cannot migrate
\end{itemize}

\subsubsection{AMT Solution}

Remove \texttt{serviceEndpoint} from DIDDocument entirely.

Instead:
\begin{itemize}
  \item DID lifetime: Long-term (public-key-based, years to decades)
  \item Endpoint lifetime: Session-scoped (minutes to hours)
  \item Endpoint management: Service provider responsibility only
  \item Individual burden: Zero
\end{itemize}

\subsubsection{Implementation Reality}

\begin{itemize}
  \item Wallet is invoked by service app (already knows its own endpoint)
  \item DIDDocument contains only public key for cryptographic verification
  \item Endpoint information implicit in application context
  \item No distributed endpoint discovery needed
\end{itemize}

For complete did:amt specification details, see Chapter 2 (did:amt Method Specification).

\section{DIDComm Integration}

AMATELUS employs DIDComm Messaging v2.1 as the sole communication protocol.

\subsection{Message Structure}

\begin{verbatim}
DIDCommMessageSend := {
  senderDID: ValidDID,
  senderDoc: Option<ValidDIDDocument>,
  vcs: List<ValidVC>,
  zkp: Option<ValidZKP>
}
\end{verbatim}

\subsection{Security Properties}

\begin{itemize}
  \item \textbf{ECDH-1PU authenticated encryption}: Sender identity authenticated exclusively to recipient
  \item \textbf{Message-level security}: Independent of transport layer
  \item \textbf{Transport agnosticism}: Works over HTTPS, BLE, WebSocket, etc.
  \item \textbf{Sender anonymity (optional)}: Anoncrypt layer available if needed
\end{itemize}

\section{Verifiable Credentials}

\subsection{VC Structure}

\begin{verbatim}
VC := {
  issuer: DID_issuer,
  subject: DID_subject,
  claims: Claims,
  signature: Signature,
  credentialStatus: RevocationInfo,
  deLinkageInfo: Option<DeLinkageInfo>
}
\end{verbatim}

\subsection{Delinkage Information}

To prevent cross-service correlation:

\begin{verbatim}
DeLinkageInfo := {
  identityDID: DID,        // Long-term identity DID
  communicationDID: DID    // Session-specific DID
}
\end{verbatim}

ZKP proves ownership of both DIDs without revealing either to external parties.

\section{Zero-Knowledge Proofs}

\subsection{Dual Nonce Binding}

\begin{verbatim}
ZKP := (π, x, nonce_holder, nonce_verifier)
where π proves knowledge of w such that
R(x, w, nonce_holder, nonce_verifier) = 1
\end{verbatim}

\subsection{Nonce Generation and Binding}

Both Holder and Verifier generate independent nonces (≥128 bits):

\begin{enumerate}
  \item Verifier generates \texttt{nonce2} (challenge)
  \item Verifier sends \texttt{nonce2} to Holder
  \item Holder generates \texttt{nonce1} independently
  \item Holder combines: \texttt{nonce\_combined = H(nonce1 || nonce2)}
  \item ZKP bound to \texttt{nonce\_combined}
  \item Holder returns (ZKP, nonce1) to Verifier
  \item Verifier verifies both nonces cryptographically
\end{enumerate}

\subsection{Security Property: Mutual Defense}

If either party generates unique nonces:
\begin{itemize}
  \item Holder bug (fixed nonce1) $\Rightarrow$ Verifier's nonce2 provides protection
  \item Verifier bug (fixed nonce2) $\Rightarrow$ Holder's nonce1 provides protection
  \item Both unique $\Rightarrow$ Complete protection from replay
\end{itemize}

Neither party depends solely on the other's implementation quality.

\subsection{Computational Efficiency}

\begin{itemize}
  \item \textbf{Offline precomputation}: Heavy circuit evaluation (minutes to hours)
  \item \textbf{Real-time nonce binding}: Light operations only (milliseconds to hundreds of milliseconds)
  \item \textbf{UX compatibility}: User interaction completes within tolerance (3 seconds)
\end{itemize}

\chapter{Trust Architecture}

\section{Responsibility Boundaries}

\begin{center}
\begin{tabular}{|l|c|c|}
\hline
\textbf{Component} & \textbf{AMATELUS} & \textbf{Service Provider} \\
\hline
Public Key Infrastructure & \checkmark & -- \\
DID generation & \checkmark & -- \\
VC issuance/validation (1-layer) & \checkmark & Policy \\
ZKP generation/verification & \checkmark & -- \\
Endpoint management & -- & \checkmark \\
Message delivery & -- & \checkmark \\
Authorization decisions & -- & \checkmark \\
Communication security (TLS) & -- & \checkmark \\
\hline
\end{tabular}
\end{center}

\section{Trust Origin}

\begin{itemize}
  \item \textbf{Cryptographic trust}: Originating from AMATELUS protocol
  \item \textbf{Operational trust}: Originating from service provider (centralized)
  \item \textbf{Authorization trust}: Originating from service provider (centralized)
\end{itemize}

The separation prevents AMATELUS from assuming responsibilities it cannot scale to manage globally.

\section{One-Layer Trust Limitation}

AMATELUS validates only 1-layer VC chains:

\begin{itemize}
  \item \textbf{0-layer}: Direct issuance from trusted anchor
  \item \textbf{1-layer}: Delegated issuance (trustee validated against anchor)
  \item \textbf{2+ layers}: Explicitly \emph{not} validated by AMATELUS protocol
\end{itemize}

This prevents:
\begin{itemize}
  \item Delegation chain attacks
  \item Circular credential verification
  \item Unbounded revocation propagation
\end{itemize}

\chapter{Cryptographic Foundations}

\section{Security Assumptions}

\begin{itemize}
  \item \textbf{Collision-resistant hash}: SHA3-512 provides 128-bit security against quantum adversaries
  \item \textbf{Unforgeable signatures}: Dilithium2 provides 128-bit security against quantum adversaries
  \item \textbf{ZKP soundness}: Standard zero-knowledge properties (completeness, soundness, zero-knowledge)
  \item \textbf{Independent nonce generation}: Cryptographically random nonces with sufficient entropy
\end{itemize}

\section{Threat Model and Mitigations}

\subsection{Impersonation Attack (Different Secret Key)}

\begin{itemize}
  \item \textbf{Attack}: Attacker uses different secret key to forge ZKP
  \item \textbf{Defense}: DIDComm makes sender's public key known to recipient
  \item \textbf{Result}: Different secret key $\Rightarrow$ signature verification fails
  \item \textbf{Protocol layer}: Cryptographically prevented
\end{itemize}

\subsection{Replay Attack (Same ZKP, Same User)}

\begin{itemize}
  \item \textbf{Attack}: Legitimate user reuses identical ZKP in new session
  \item \textbf{Defense}: Dual nonce binding ensures each ZKP corresponds to unique nonce pair
  \item \textbf{Result}: Session-specific ZKP cannot be reused
  \item \textbf{Service layer}: Optional nonce history management (service responsibility)
\end{itemize}

\subsection{Man-in-the-Middle Attack}

\begin{itemize}
  \item \textbf{Attack}: Intermediary intercepts and modifies messages
  \item \textbf{Defense}: ECDH-1PU authenticated encryption (service provider responsibility)
  \item \textbf{Result}: Message authentication prevents undetected modification
  \item \textbf{Transport layer}: TLS/HTTPS (service provider responsibility)
\end{itemize}

\subsection{Sybil Attack (Multiple DIDs)}

\begin{itemize}
  \item \textbf{Attack}: Single entity controls multiple DIDs to gain asymmetric advantages
  \item \textbf{Protocol position}: Multiple DID possession is intentional design for privacy
  \item \textbf{Mitigation}: Anonymous Hash Identifier (AHI) restricts per-audit-domain abuse
  \item \textbf{Social layer}: Operational policy and service-specific restrictions
\end{itemize}

\chapter{Privacy Architecture}

\section{Anti-Linkability Across Services}

Multiple DIDs enable cross-service unlinkability:

\begin{verbatim}
∀ DID₁, DID₂, Service₁, Service₂:
  (Service₁ ≠ Service₂) ∧
  Link(DID₁, DID₂) requires 2^128 quantum operations
\end{verbatim}

\section{Zero-Knowledge Property}

ZKP reveal attribute ownership without revealing identity:

\begin{itemize}
  \item \textbf{Public}: Attribute claimed (age ≥ 18)
  \item \textbf{Hidden}: Identity proving the attribute
  \item \textbf{Hidden}: Secret key generating the proof
  \item \textbf{Verified}: ZKP authenticity via public key
\end{itemize}

\section{Deniable Authentication}

For privacy-sensitive scenarios, anonymous encryption (Anoncrypt) available:

\begin{itemize}
  \item Sender identity hidden from intermediaries
  \item Recipient verifies proof authenticity (still authenticated)
  \item Sender maintains plausible deniability
\end{itemize}

\chapter{Implementation Guidance}

\section{Wallet Implementation Requirements}

\subsection{Secret Key Management}

\begin{itemize}
  \item Store private keys in secure enclave (iOS/Android hardware)
  \item Never export unencrypted private keys
  \item Implement key derivation from user-memorizable seeds (BIP39 compatible)
\end{itemize}

\subsection{DID Lifecycle}

\subsubsection{Identity DIDs}

\begin{itemize}
  \item Generated once per credential lifetime (typically 1-10 years)
  \item Persisted across wallet updates and device changes
  \item Associated with formal credentials (passport, license, etc.)
\end{itemize}

\subsubsection{Communication DIDs}

\begin{itemize}
  \item Generated per service engagement or per session (configurable)
  \item Destroyed if VC not issued (one-time verification)
  \item Persisted if VC issued (login reuse case)
  \item Linked to counterparty DID in wallet for session continuation
\end{itemize}

\subsection{VC Storage}

\begin{itemize}
  \item Store issued VCs with \texttt{deLinkageInfo}
  \item Enable offline VC verification
  \item Implement selective disclosure (claim subset presentation)
  \item Track VC issuance context (service, timestamp, nonce)
\end{itemize}

\section{Service Provider Integration}

\subsection{DIDComm Endpoint Setup}

\begin{enumerate}
  \item Publish service endpoint URL (HTTPS with valid TLS)
  \item Accept POST requests with DIDComm messages
  \item Parse and cryptographically verify ZKP
  \item Execute service logic based on verified claims
  \item Return VC (if registration service) or result (if verification-only)
\end{enumerate}

\subsection{Authorization Decision Points}

Services must decide:

\begin{itemize}
  \item Which claims are required for service access
  \item Which DIDs are trusted for VC validation
  \item How to handle multiple DIDs from same user (user linking policy)
  \item Whether to issue persistent VC or one-time verification
\end{itemize}

\subsection{Nonce Management (Optional Service Feature)}

If replay prevention beyond DIDComm is required:

\begin{enumerate}
  \item Generate unique nonce per session
  \item Send nonce to wallet
  \item Verify returned ZKP includes nonce in cryptographic binding
  \item Record used nonces to prevent reuse
  \item Clean up nonce history based on session TTL
\end{enumerate}

Note: Nonce management is \emph{optional} at service layer, not protocol-required.

\section{Deployment Scenarios}

\subsection{One-Time Verification (Age Confirmation)}

\begin{itemize}
  \item Holder generates ephemeral communication DID
  \item Generates ZKP proving age ≥ 18
  \item Verifier validates ZKP, permits or denies access
  \item Communication DID destroyed (no VC issued)
  \item No persistent state
\end{itemize}

\subsection{Service Registration (Account Opening)}

\begin{itemize}
  \item Holder generates communication DID (persistent)
  \item Generates ZKP proving identity claims
  \item Issuer validates, creates account
  \item Issuer generates VC (e.g., account holder credential)
  \item Holder stores VC and communication DID-to-service association
  \item Later: Login reuses same communication DID
\end{itemize}

\subsection{Continuous Authentication (SNS, Messaging)}

\begin{itemize}
  \item Holder generates ephemeral communication DID per login
  \item Generates ZKP proving phone number ownership
  \item Service validates, establishes session
  \item Session token replaces ZKP for subsequent requests
  \item Communication DID may be stored for optional reconnection hint
\end{itemize}

\chapter{Conclusion}

AMATELUS provides a clean separation of concerns:

\begin{itemize}
  \item \textbf{Cryptographic layer}: AMATELUS ensures authentication and privacy
  \item \textbf{Service layer}: Service providers ensure authorization and delivery
  \item \textbf{User layer}: Users choose which services and DIDs to use
\end{itemize}

This architecture achieves:
\begin{itemize}
  \item \textbf{Security}: Formal cryptographic guarantees
  \item \textbf{Scalability}: Linear throughput with users
  \item \textbf{Privacy}: Cross-service unlinkability
  \item \textbf{Simplicity}: Clear responsibility boundaries
  \item \textbf{Practicality}: No distributed infrastructure burden on users
\end{itemize}

The protocol enables centralized services to provide decentralized authentication—a pragmatic foundation for digital governance in the real world.
