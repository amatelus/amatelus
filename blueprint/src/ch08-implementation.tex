\chapter{Implementation Guidance}


\begin{definition}
  \label{def:impl-chapter}
  This chapter covers implementation guidance aspects of AMATELUS.
  \uses{def:amatelus,def:crypto-assumptions}
  \lean{AMATELUS.Operations}
  \leanok
\end{definition}
\section{Wallet Implementation Requirements}

\subsection{Secret Key Management}

\begin{itemize}
  \item Store private keys in secure enclave (iOS/Android hardware)
  \item Never export unencrypted private keys
  \item Implement key derivation from user-memorizable seeds (BIP39 compatible)
\end{itemize}

\subsection{DID Lifecycle}

\subsubsection{Identity DIDs}

\begin{itemize}
  \item Generated once per credential lifetime (typically 1-10 years)
  \item Persisted across wallet updates and device changes
  \item Associated with formal credentials (passport, license, etc.)
\end{itemize}

\subsubsection{Communication DIDs}

\begin{itemize}
  \item Generated per service engagement or per session (configurable)
  \item Destroyed if VC not issued (one-time verification)
  \item Persisted if VC issued (login reuse case)
  \item Linked to counterparty DID in wallet for session continuation
\end{itemize}

\subsection{VC Storage}

\begin{itemize}
  \item Store issued VCs with \texttt{deLinkageInfo}
  \item Enable offline VC verification
  \item Implement selective disclosure (claim subset presentation)
  \item Track VC issuance context (service, timestamp, nonce)
\end{itemize}

\section{Service Provider Integration}

\subsection{DIDComm Endpoint Setup}

\begin{enumerate}
  \item Publish service endpoint URL (HTTPS with valid TLS)
  \item Accept POST requests with DIDComm messages
  \item Parse and cryptographically verify ZKP
  \item Execute service logic based on verified claims
  \item Return VC (if registration service) or result (if verification-only)
\end{enumerate}

\subsection{Authorization Decision Points}

Services must decide:

\begin{itemize}
  \item Which claims are required for service access
  \item Which DIDs are trusted for VC validation
  \item How to handle multiple DIDs from same user (user linking policy)
  \item Whether to issue persistent VC or one-time verification
\end{itemize}

\subsection{Nonce Management (Optional Service Feature)}

If replay prevention beyond DIDComm is required:

\begin{enumerate}
  \item Generate unique nonce per session
  \item Send nonce to wallet
  \item Verify returned ZKP includes nonce in cryptographic binding
  \item Record used nonces to prevent reuse
  \item Clean up nonce history based on session TTL
\end{enumerate}

Note: Nonce management is \emph{optional} at service layer, not protocol-required.

\section{Deployment Scenarios}

\subsection{One-Time Verification (Age Confirmation)}

\begin{itemize}
  \item Holder generates ephemeral communication DID
  \item Generates ZKP proving age $\geq$ 18
  \item Verifier validates ZKP, permits or denies access
  \item Communication DID destroyed (no VC issued)
  \item No persistent state
\end{itemize}

\subsection{Service Registration (Account Opening)}

\begin{itemize}
  \item Holder generates communication DID (persistent)
  \item Generates ZKP proving identity claims
  \item Issuer validates, creates account
  \item Issuer generates VC (e.g., account holder credential)
  \item Holder stores VC and communication DID-to-service association
  \item Later: Login reuses same communication DID
\end{itemize}

\subsection{Continuous Authentication (SNS, Messaging)}

\begin{itemize}
  \item Holder generates ephemeral communication DID per login
  \item Generates ZKP proving phone number ownership
  \item Service validates, establishes session
  \item Session token replaces ZKP for subsequent requests
  \item Communication DID may be stored for optional reconnection hint
\end{itemize}

