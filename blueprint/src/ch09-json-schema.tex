\chapter{Formally Verifiable JSON Schema Subset}

\section{Abstract}

The AMATELUS protocol employs a formally verifiable subset of JSON Schema 2020-12 for Verifiable Credentials validation. This subset is specifically designed to enable formal verification in Lean 4 without axioms or partial definitions.

\section{Design Principles}

The JSON Schema subset is built on these principles:

\begin{itemize}
  \item \textbf{Formal Verification}: All features MUST be provably terminating in Lean 4
  \item \textbf{Protocol Guarantee}: Only features in this subset are guaranteed by AMATELUS protocol
  \item \textbf{Implementation Freedom}: Wallets MAY support full JSON Schema 2020-12 beyond protocol guarantees
  \item \textbf{Deterministic Validation}: Same schema produces identical results across all implementations
  \item \textbf{Future Evolution}: As Lean capability increases, subset expands toward full JSON Schema 2020-12
\end{itemize}

\section{Formalization Strategy}

\subsection{Why a Subset?}

Full JSON Schema 2020-12 includes features that prevent formal verification:

\begin{itemize}
  \item \textbf{Circular references} (via \texttt{\$ref}, \texttt{\$defs}): Cause non-termination
  \item \textbf{Dynamic property validation} (\texttt{additionalProperties}, \texttt{patternProperties}): Require unbounded iteration
  \item \textbf{Infinite nesting}: Composition keywords can nest arbitrarily
  \item \textbf{Format annotations}: Depend on external specifications
\end{itemize}

The subset eliminates these features to enable formal verification.

\subsection{Lean Formalization Structure}

\begin{verbatim}
inductive SchemaKeyword
  | type : List String -> SchemaKeyword
  | maxLength : Nat -> SchemaKeyword
  | minLength : Nat -> SchemaKeyword
  | pattern : String -> SchemaKeyword
  | maximum : Float -> SchemaKeyword
  | minimum : Float -> SchemaKeyword
  | multipleOf : Float -> SchemaKeyword
  | maxItems : Nat -> SchemaKeyword
  | minItems : Nat -> SchemaKeyword
  | uniqueItems : Bool -> SchemaKeyword
  | items : Schema -> SchemaKeyword
  | maxProperties : Nat -> SchemaKeyword
  | minProperties : Nat -> SchemaKeyword
  | required : List String -> SchemaKeyword
  | properties : List (String × Schema) -> SchemaKeyword
  | enum : List JSONValue -> SchemaKeyword
  | const : JSONValue -> SchemaKeyword
  | allOf : List Schema -> SchemaKeyword
  | anyOf : List Schema -> SchemaKeyword
  | oneOf : List Schema -> SchemaKeyword
  | not : Schema -> SchemaKeyword
  | title : String -> SchemaKeyword
  | description : String -> SchemaKeyword
\end{verbatim}

Note: Excluded keywords like \texttt{\$ref}, \texttt{additionalProperties}, \texttt{format} are not in the inductive type.

\subsection{Termination Guarantee}

All supported features are \emph{provably terminating}:

\begin{enumerate}
  \item \textbf{No circular references}: \texttt{\$ref} excluded entirely
  \item \textbf{Bounded recursion}: Composition nesting limited to 3 levels
  \item \textbf{Finite structure}: All schemas have finite size
\end{enumerate}

\section{Supported Features}

The following JSON Schema keywords are \textbf{GUARANTEED} by AMATELUS protocol:

\subsection{Type System}

\begin{itemize}
  \item \textbf{\texttt{type}}: String or array of strings (null, boolean, object, array, number, string, integer)
\end{itemize}

\subsection{String Validation}

\begin{itemize}
  \item \textbf{\texttt{maxLength}}: Non-negative integer
  \item \textbf{\texttt{minLength}}: Non-negative integer
  \item \textbf{\texttt{pattern}}: ECMA-262 regular expression (abstracted in Lean)
\end{itemize}

\subsection{Numeric Validation}

\begin{itemize}
  \item \textbf{\texttt{maximum}}: Inclusive upper bound
  \item \textbf{\texttt{minimum}}: Inclusive lower bound
  \item \textbf{\texttt{multipleOf}}: Value must be multiple of given number
\end{itemize}

\subsection{Array Validation}

\begin{itemize}
  \item \textbf{\texttt{maxItems}}: Non-negative integer
  \item \textbf{\texttt{minItems}}: Non-negative integer
  \item \textbf{\texttt{uniqueItems}}: Boolean (default: false)
  \item \textbf{\texttt{items}}: Schema applied to all array items
\end{itemize}

\subsection{Object Validation}

\begin{itemize}
  \item \textbf{\texttt{maxProperties}}: Non-negative integer
  \item \textbf{\texttt{minProperties}}: Non-negative integer
  \item \textbf{\texttt{required}}: Array of unique property names that must exist
  \item \textbf{\texttt{properties}}: Object mapping property names to schemas
\end{itemize}

\subsection{Generic Validation}

\begin{itemize}
  \item \textbf{\texttt{enum}}: Non-empty array of allowed values
  \item \textbf{\texttt{const}}: Single fixed JSON value
\end{itemize}

\subsection{Schema Composition (Depth-Limited)}

\begin{itemize}
  \item \textbf{\texttt{allOf}}: Instance must validate against ALL subschemas (max 3 levels)
  \item \textbf{\texttt{anyOf}}: Instance must validate against AT LEAST ONE subschema (max 3 levels)
  \item \textbf{\texttt{oneOf}}: Instance must validate against EXACTLY ONE subschema (max 3 levels)
  \item \textbf{\texttt{not}}: Instance must NOT validate against subschema (max 3 levels)
\end{itemize}

\subsection{Annotation Keywords}

\begin{itemize}
  \item \textbf{\texttt{title}}: Human-readable title (informational only)
  \item \textbf{\texttt{description}}: Detailed description (informational only)
\end{itemize}

\section{Excluded Features}

The following JSON Schema features are \textbf{NOT GUARANTEED} at protocol level:

\subsection{Reference Keywords}

\begin{itemize}
  \item \textbf{\texttt{\$ref}}: Schema references (prevents termination proof)
  \item \textbf{\texttt{\$defs}}: Schema definitions (only useful with \texttt{\$ref})
  \item \textbf{\texttt{\$dynamicRef}}, \textbf{\texttt{\$dynamicAnchor}}: Dynamic references
\end{itemize}

\subsection{Dynamic Property Validation}

\begin{itemize}
  \item \textbf{\texttt{additionalProperties}}: Ignored at protocol level (wallets MAY validate)
  \item \textbf{\texttt{patternProperties}}: Not supported
  \item \textbf{\texttt{propertyNames}}: Not supported
  \item \textbf{\texttt{unevaluatedProperties}}, \textbf{\texttt{unevaluatedItems}}: Not supported
\end{itemize}

\subsection{Other Excluded Keywords}

\begin{itemize}
  \item \textbf{\texttt{if}}, \textbf{\texttt{then}}, \textbf{\texttt{else}}: Conditional schemas (use \texttt{oneOf} instead)
  \item \textbf{\texttt{prefixItems}}, \textbf{\texttt{contains}}: Advanced array features
  \item \textbf{\texttt{dependentRequired}}, \textbf{\texttt{dependentSchemas}}: Dependency keywords
  \item \textbf{\texttt{exclusiveMaximum}}, \textbf{\texttt{exclusiveMinimum}}: Exclusive bounds
  \item \textbf{\texttt{format}}: Format annotations (use \texttt{pattern} instead)
\end{itemize}

\section{Conformance Requirements}

\subsection{Conformant Schema}

A schema is AMATELUS-conformant if:

\begin{enumerate}
  \item Contains ONLY keywords from supported features
  \item Composition nesting does NOT exceed 3 levels
  \item \texttt{required} array contains unique strings
  \item \texttt{enum} array is non-empty
  \item Numeric constraints are valid numbers
  \item Size constraints are non-negative integers
  \item \texttt{multipleOf} is greater than 0
\end{enumerate}

\subsection{Conformant Validator}

An AMATELUS-conformant validator MUST:

\begin{enumerate}
  \item Validate all supported keywords correctly
  \item \textbf{Ignore} excluded keywords without error (e.g., \texttt{additionalProperties})
  \item Enforce nesting depth limit (3 levels)
  \item Produce deterministic results
  \item Reject schemas exceeding nesting limits
\end{enumerate}

\section{Future Evolution: Convergence Toward Full JSON Schema}

\subsection{Current Status}

This subset represents a \emph{formally verifiable foundation}. As Lean's formalization capabilities advance, the subset will gradually expand toward full JSON Schema 2020-12 compliance.

\subsection{Formalization-Driven Expansion}

The expansion strategy follows a principle: \textbf{expand the subset only as new features become provably terminating in Lean}.

\begin{itemize}
  \item \textbf{Lean 4 v1.0 onwards}: Recursive reference support (with termination metrics) may enable \texttt{\$ref} formalization
  \item \textbf{Future versions}: Advanced pattern matching and dependent types may enable:
    \begin{itemize}
      \item Dynamic property validation with bounded iteration proofs
      \item Conditional schema validation
      \item More complex nesting patterns
    \end{itemize}
  \item \textbf{Goal}: Eventually, full JSON Schema 2020-12 becomes formally verifiable
\end{itemize}

\subsection{Backward Compatibility}

\begin{itemize}
  \item Current AMATELUS-conformant schemas remain valid
  \item New features are only added when formally verified
  \item Protocol versioning ensures interoperability during transitions
  \item Wallets always free to support full JSON Schema 2020-12 beyond protocol guarantees
\end{itemize}

\subsection{Versioning Strategy}

\begin{verbatim}
AMATELUS JSON Schema Subset v1.0 (current)
- Formally verified features: 20+ keywords
- Excluded features: circular refs, dynamic validation, format

AMATELUS JSON Schema Subset v1.1 (hypothetical)
- New formalization: conditional validation via oneOf expansion
- Result: if/then/else pattern becomes recommended practice

AMATELUS JSON Schema Subset v2.0 (future)
- Lean capability: recursive references with termination proof
- Result: $ref support added with formal verification guarantees
\end{verbatim}

\section{Example: AMATELUS-Conformant VC Schema}

\begin{verbatim}
{
  "title": "Identity Credential",
  "type": "object",
  "properties": {
    "type": {
      "type": "string",
      "enum": ["IdentityCredential"]
    },
    "credentialSubject": {
      "type": "object",
      "properties": {
        "givenName": {
          "type": "string",
          "minLength": 1,
          "maxLength": 100
        },
        "familyName": {
          "type": "string",
          "minLength": 1,
          "maxLength": 100
        },
        "birthDate": {
          "type": "string",
          "pattern": "^\\d{4}-\\d{2}-\\d{2}$"
        }
      },
      "required": ["givenName", "familyName"],
      "minProperties": 2,
      "maxProperties": 5
    }
  },
  "required": ["type", "credentialSubject"]
}
\end{verbatim}

\subsection{Why This Schema is Conformant}

\begin{itemize}
  \item Uses only supported keywords (\texttt{type}, \texttt{properties}, \texttt{required}, \texttt{enum}, \texttt{pattern})
  \item Nesting depth is 2 levels (within limit of 3)
  \item All constraints are valid (minLength, maxLength, minProperties, maxProperties)
  \item No circular references or dynamic validation
  \item Formally verifiable in Lean 4
\end{itemize}

\section{Security and Robustness}

\subsection{Validation Guarantees}

\begin{itemize}
  \item \textbf{Termination}: All schemas guaranteed to validate in finite time
  \item \textbf{Determinism}: Same input always produces identical output
  \item \textbf{No side effects}: Validation is pure computation
\end{itemize}

\subsection{Regular Expression Safety}

\begin{itemize}
  \item \textbf{ReDoS prevention}: Implementation SHOULD enforce regex timeout limits
  \item \textbf{Abstraction in Lean}: Regex matching abstracted as \texttt{MatchedString}/\texttt{UnmatchedString}
  \item \textbf{Implementation responsibility}: Wallet implementations must validate regex syntax
\end{itemize}

\subsection{Schema Complexity Limits}

\begin{itemize}
  \item \textbf{Max nesting}: 3 levels prevents deeply nested schemas
  \item \textbf{Max properties}: No explicit limit, but implementations MAY enforce reasonable bounds
  \item \textbf{No recursion}: Eliminates infinite loops and stack overflow risks
\end{itemize}

