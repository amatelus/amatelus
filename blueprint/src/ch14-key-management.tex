% AMATELUS Secret Key Lifecycle Management Specification
% Based on version 1.0 dated 2025-10-14
% Covers emergency scenarios: theft, loss, rotation, death, guardianship

\chapter{Secret Key Lifecycle Management}

\section{Overview}

This chapter specifies the management of secret keys in AMATELUS, particularly emergency procedures for:
device theft, loss, damage, device replacement, account rotation, death, and guardianship.

\subsection{Key Principles}

\begin{enumerate}
  \item \textbf{Determinism Principle}: Only operations with secret key signatures are cryptographically deterministic
  \item \textbf{Contactability Limit}: Only reachable Issuers can respond; unreachable Issuers cannot help
  \item \textbf{Mathematical Safety Limit}: Safe recovery of DIDs without identity credential linkage is impossible
\end{enumerate}

\subsection{Scenario Classification}

Secret key issues fall into five categories:

\begin{enumerate}
  \item \textbf{Leakage}: Secret key in hand, possibly compromised to third party
  \item \textbf{Loss with Abuse Concern}: Suspected theft or compromise (backup available)
  \item \textbf{Loss without Abuse Concern}: Device damage/loss without compromise risk
  \item \textbf{Update}: Planned key rotation (both old and new keys available)
  \item \textbf{Death}: Permanent loss of physical access
  \item \textbf{Guardianship}: Capacity-restricted individual under legal guardianship
\end{enumerate}

\section{Secret Key Leakage Response Flow}

\subsection{Preconditions}

\begin{itemize}
  \item \textbf{Old secret key available}: Holder still possesses possibly-compromised key
  \item \textbf{Signature operations possible}: Can cryptographically sign revocation requests
  \item \textbf{Third party compromise suspected}: Key copy may exist with attackers
\end{itemize}

\subsection{Response Procedure}

\subsubsection{Step 1: Emergency Response Initiation}

\begin{enumerate}
  \item Holder generates new secret key pair $(newDID, newSK)$
  \item Holder creates revocation signature using old key:
    \begin{itemize}
      \item $signatureByOldKey = \text{Sign}(oldSK, H(oldDID \parallel newDID \parallel timestamp))$
    \end{itemize}
\end{enumerate}

\subsubsection{Step 2: Revocation Request to All Issuers}

Holder sends to \textbf{all Issuers} that issued VCs:

\begin{lstlisting}[language=json]
{
  "requestType": "revocationDueToLeakage",
  "oldDID": "did:amatelus:old123...",
  "newDID": "did:amatelus:new456...",
  "timestamp": "2025-10-14T10:30:00Z",
  "revocationReason": "Secret key leakage suspected",
  "signatureByOldKey": "signature(oldSK, hash(oldDID || newDID || timestamp))"
}
\end{lstlisting}

Cryptographic guarantee:
\begin{itemize}
  \item \texttt{signatureByOldKey} cryptographically proves rightful old key owner
  \item Third party cannot forge signature (Dilithium2 security)
\end{itemize}

\subsubsection{Step 3: Issuer Processing}

Each Issuer executes:

\begin{enumerate}
  \item Verify \texttt{signatureByOldKey} (reject if verification fails)
  \item Add all VCs issued for \texttt{oldDID} to revocation list
  \item Update and publish new Merkle Root
  \item Permanently reject new VC requests from \texttt{oldDID}
  \item (Optional) Re-issue VC for \texttt{newDID} following policy
\end{enumerate}

\subsubsection{Step 4: Revocation Completion Verification}

Holder confirms:

\begin{enumerate}
  \item Check each Issuer's MerkleRevocationList (VCs revoked)
  \item List unreachable Issuers (unrevoked VCs remain)
\end{enumerate}

\subsubsection{Timeline}

\begin{verbatim}
T=0:      Leakage detection
T=+5min:  New key generation, revocation signature
T=+30min: Revocation requests sent to all Issuers
T=+1h:    Issuer revocation processing complete (target SLA)
T=+24h:   Revocation confirmation, re-issuance begins
\end{verbatim}

\subsubsection{Security Characteristics}

\begin{table}[h]
\centering
\begin{tabular}{|l|l|l|}
\hline
\textbf{Property} & \textbf{Guarantee Level} & \textbf{Basis} \\
\hline
Old VC revocation & \textbf{Deterministic} & Cryptographic signature \\
Old DID rejection & \textbf{Deterministic} & Issuer blacklist \\
Unreachable Issuers & \textbf{None} & No contact possible \\
Third-party misuse & \textbf{Time-dependent} & Issuer revocation speed \\
\hline
\end{tabular}
\end{table}

\section{Secret Key Loss: Abuse Concern}

\subsection{Scenario}

\begin{itemize}
  \item Device theft, malware infection, or compromise suspected
  \item Old secret key unavailable, but backup recovery possible
\end{itemize}

\subsection{Response}

If backup recovery possible: Execute Section 3 (leakage response)

If no backup: Proceed to Section 5 (trust inheritance flow)

\section{Secret Key Loss: Trust Inheritance Flow}

\subsection{Preconditions}

\begin{itemize}
  \item Old secret key unavailable (device lost/damaged)
  \item No abuse concern (device not stolen)
  \item New secret key pair generated
\end{itemize}

\subsection{Case A: Identity Credential Available}

\subsubsection{Trust Inheritance Principle}

\begin{verbatim}
Old DID (oldDID)
  └─ Identity VC (e.g., driver's license VC)
      └─ Issuer = Police/Municipality
      └─ Linked to real-world identity document

↓ Trust Inheritance

New DID (newDID)
  └─ DID Migration VC (DIDMigrationVC)
      └─ Issuer = Police/Municipality
      └─ Proves same-person relationship
\end{verbatim}

\subsubsection{Response Procedure}

\textbf{Step 1: Real-World Identity Verification at Municipality}

Holder presents:
\begin{enumerate}
  \item Physical identity document (driver's license, national ID card, etc.)
  \item New DID (newDID)
  \item Old DID information if available
\end{enumerate}

\textbf{Step 2: Issuer Record Verification}

Issuer checks:
\begin{enumerate}
  \item Physical identity document authenticity
  \item Past VC issuance records
  \item Link between document and oldDID
\end{enumerate}

\textbf{Step 3: DID Migration VC Issuance}

\begin{lstlisting}[language=json]
{
  "@context": ["https://www.w3.org/2018/credentials/v1"],
  "type": ["VerifiableCredential", "DIDMigrationCredential"],
  "issuer": "did:amatelus:police-station-tokyo",
  "issuanceDate": "2025-10-14T11:00:00Z",
  "credentialSubject": {
    "id": "did:amatelus:new456...",
    "oldDID": "did:amatelus:old123...",
    "migrationType": "trustInheritance",
    "migrationReason": "Device damage causing key loss",
    "verificationMethod": "Physical identity document verification",
    "identityDocument": {
      "type": "DriversLicense",
      "documentNumber": "123456789012",
      "issuerName": "Tokyo Public Safety Commission",
      "verifiedAt": "2025-10-14T10:30:00Z"
    }
  },
  "proof": {
    "type": "Dilithium2Signature2025",
    "verificationMethod": "did:amatelus:police-station-tokyo#key-1"
  }
}
\end{lstlisting}

\textbf{Step 4: Re-issuance from Other Issuers}

Holder presents to other Issuers:
\begin{enumerate}
  \item DID Migration VC
  \item New ZKP generated with newDID
\end{enumerate}

Issuer judgment: Accept re-issuance following policy

\subsubsection{Security Characteristics}

\begin{table}[h]
\centering
\begin{tabular}{|l|l|}
\hline
\textbf{Property} & \textbf{Level} \\
\hline
Identity proof & High (physical document verification) \\
DID migration legitimacy & High (trusted Issuer proof) \\
Other Issuers acceptance & Policy-dependent \\
Old VC revocation & Impossible (no old key) \\
\hline
\end{tabular}
\end{table}

\subsection{Case B: No Identity Credential}

\subsubsection{Problem}

\begin{itemize}
  \item No cryptographic signature capability (old key lost)
  \item No real-world identity link in system
  \item Issuer cannot verify ``who'' was the oldDID owner
\end{itemize}

\textbf{Mathematical Safety Limit Theorem}:

\begin{quote}
\textit{Safe recovery of DIDs without identity credential linkage is information-theoretically impossible.}
\end{quote}

\subsubsection{Issuer Policy Examples}

Each Issuer must develop independent policy:

\textbf{University Transcript VC}:
\begin{itemize}
  \item Present student ID at window
  \item Photo identification
  \item Cross-reference past issued transcripts
  \item Re-issue for new DID
\end{itemize}

\textbf{Online Service Member VC}:
\begin{itemize}
  \item Confirmation code to registered email
  \item Security questions
  \item Cross-reference usage history
  \item Re-issue for new DID
\end{itemize}

\textbf{Anonymous Service VC}:
\begin{itemize}
  \item No identity verification possible
  \item Re-issuance impossible
  \item Re-register as new user
\end{itemize}

\subsubsection{Practical Conclusion}

\begin{quote}
DIDs without identity-credential linkage are effectively unrecoverable if lost.

\textbf{Preventive strategies}:
\begin{enumerate}
  \item Always obtain identity credential first
  \item Maintain secret key backups in multiple locations
  \item Use multi-device support
  \item Perform periodic key updates
\end{enumerate}
\end{quote}

\section{Planned Secret Key Update Flow}

\subsection{Preconditions}

\begin{itemize}
  \item Both old and new secret keys available
  \item Both key operations possible
  \item Planned update (not emergency)
\end{itemize}

\subsection{Use Cases}

\begin{enumerate}
  \item Device replacement (new smartphone)
  \item Multi-device support (PC and smartphone)
  \item Periodic key rotation (security policy)
  \item Backup device preparation
\end{enumerate}

\subsection{Procedure}

\subsubsection{Step 1: Dual-Key Relation Proof}

\begin{enumerate}
  \item Generate new secret key pair $(newDID, newSK)$
  \item Create dual-signature relation proof:
    \begin{itemize}
      \item $sig_{old} = \text{Sign}(oldSK, H(oldDID \parallel newDID \parallel timestamp))$
      \item $sig_{new} = \text{Sign}(newSK, H(newDID \parallel oldDID \parallel timestamp))$
    \end{itemize}
\end{enumerate}

\subsubsection{Step 2: DID Update Notification}

\begin{lstlisting}[language=json]
{
  "requestType": "didUpdate",
  "oldDID": "did:amatelus:old123...",
  "newDID": "did:amatelus:new456...",
  "timestamp": "2025-10-14T10:30:00Z",
  "updateReason": "Device replacement",
  "transitionPeriod": "30days",
  "signatureByOldKey": "...",
  "signatureByNewKey": "...",
  "requestedAction": "issueNewVCWithNewDID"
}
\end{lstlisting}

Cryptographic guarantee:
\begin{itemize}
  \item Both signatures prove \texttt{oldDID} and \texttt{newDID} owner is same person
\end{itemize}

\subsubsection{Step 3: Issuer Processing (Flexible Options)}

\textbf{Option A}: Issue new VC with new DID
\begin{itemize}
  \item New VC with \texttt{newDID} as subject
  \item Keep old VC valid
\end{itemize}

\textbf{Option B}: Extend old VC validity
\begin{itemize}
  \item Set transition period
  \item Next update issues new VC
\end{itemize}

\textbf{Option C}: Issue DID Update Credential
\begin{itemize}
  \item Prove \texttt{oldDID} to \texttt{newDID} relation
  \item Allow other Verifiers to verify link
\end{itemize}

\subsubsection{Step 4: Transition Period Management}

\begin{verbatim}
Transition period (e.g., 30 days):
- Both old and new DIDs valid
- Holder can use either

After transition:
- Old DID VC revoked (optional)
- New DID primary
\end{verbatim}

\subsubsection{Security Characteristics}

\begin{table}[h]
\centering
\begin{tabular}{|l|l|}
\hline
\textbf{Property} & \textbf{Level} \\
\hline
Old-new DID relation & Deterministic (dual signature) \\
New VC issuance & Deterministic (Issuer signature) \\
Old VC continuation & Issuer policy-dependent \\
Multi-device support & Possible (device-specific DIDs) \\
\hline
\end{tabular}
\end{table}

\subsection{Multi-Device Scenario}

\textbf{Recommendation}: Each device gets distinct DID

\begin{verbatim}
Device A (Smartphone):
  DID: did:amatelus:user123-mobile
  SK: SK_mobile

Device B (PC):
  DID: did:amatelus:user123-pc
  SK: SK_pc

Approach:
1. Assign unique DID per device
2. Use Section 5 procedure for each device
3. Request VCs from each Issuer for both DIDs
4. Identify with Issuer: "Same person, different devices"
5. Dual-signature proves same-person relationship
\end{verbatim}

\section{Death of Holder}

\subsection{Key Characteristics}

\begin{itemize}
  \item Secret key permanently inaccessible
  \item Misuse risk essentially zero (key cannot be used)
  \item Procedure depends on family status
  \item VC revocation incomplete but damages minimal
\end{itemize}

\subsection{Case A: No Family}

Municipal automatic processing:

\begin{enumerate}
  \item Death registration at municipality
  \item Link with VC issuance records via identity document
  \item Add holder's DIDs to municipality MerkleRevocationList
  \item (Optional) Issue death certificate VC
\end{enumerate}

\subsubsection{Issuer Processing Limitations}

\begin{itemize}
  \item Municipality Issuers: Can revoke (access to death records)
  \item Private Issuers: Cannot revoke (no access to death records)
\end{itemize}

\textbf{Security Impact}:
\begin{itemize}
  \item Private Issuer VCs remain valid
  \item But key is permanently unusable
  \item Misuse risk effectively zero
\end{itemize}

\subsection{Case B: Family Present}

Family-assisted processing:

\begin{enumerate}
  \item Municipality issues legal representative VC (LegalRepresentativeCredential)
  \item Family presents to each Issuer with representative VC
  \item Issuer processes based on policy
  \item Limited effectiveness (often not executed)
\end{enumerate}

\textbf{Security Impact}:
\begin{itemize}
  \item Some Issuers may revoke
  \item Unrevoked VCs remain
  \item Misuse essentially impossible (no key access)
  \item Damages minimal compared to leakage scenarios
\end{itemize}

\section{Guardianship (Capacity Restriction)}

\subsection{Scenario}

Holder becomes capacity-restricted (legal guardianship):
\begin{itemize}
  \item Individual is alive and retains device access
  \item Judgment capacity is legally restricted
  \item Guardian has legal authority
\end{itemize}

\textbf{Difference from death}:

\begin{verbatim}
Death:
- Secret key access: Impossible → Misuse risk zero

Guardianship:
- Secret key access: Possible (person alive) → Misuse risk exists
- Problem: Individual may sign inappropriate contracts
- Solution: Guardian can revoke/cancel contracts
\end{verbatim}

\subsection{Guardianship Initiation Flow}

\subsubsection{Family Court Guardianship Order}

\begin{enumerate}
  \item Application to family court (family member, prosecutor, etc.)
  \item Family court review of capacity
  \item Capacity evaluation
  \item Guardianship order and guardian appointment
\end{enumerate}

\subsubsection{Municipality Guardian VC Issuance}

\begin{lstlisting}[language=json]
{
  "@context": ["https://www.w3.org/2018/credentials/v1"],
  "type": ["VerifiableCredential", "GuardianCredential"],
  "issuer": "did:amatelus:city-hall-tokyo",
  "issuanceDate": "2025-10-14T14:00:00Z",
  "credentialSubject": {
    "id": "did:amatelus:guardian-789...",
    "guardianOf": "did:amatelus:ward-123...",
    "guardianType": "成年後見人",
    "authority": [
      "VC revocation request",
      "DID blacklist registration",
      "Contract cancellation",
      "Property management"
    ],
    "wardInfo": {
      "courtDecisionDate": "2025-10-01",
      "courtCaseNumber": "Reiwa 7 (Fam) No. 12345",
      "registrationNumber": "Reiwa 7 Guardianship No. 67890",
      "guardianshipStartDate": "2025-10-01"
    },
    "verificationMethod": "Court decision and registration certificate",
    "expirationDate": "2026-10-14T23:59:59Z"
  }
}
\end{lstlisting}

Key fields:
\begin{itemize}
  \item \texttt{guardianOf}: Ward's DID
  \item \texttt{authority}: Authority types
  \item \texttt{courtCaseNumber}: Legal case reference
  \item \texttt{registrationNumber}: Guardianship registry number
\end{itemize}

\subsection{Guardian VC Notification to Issuers}

Guardian notifies all Issuers:

\begin{enumerate}
  \item Present Guardian VC
  \item Request VC revocation
  \item Request DID blacklist
\end{enumerate}

Issuer processing options:

\textbf{Option A}: Complete revocation
\begin{itemize}
  \item Add to MerkleRevocationList
  \item Add DID to blacklist
\end{itemize}

\textbf{Option B}: Conditional validity
\begin{itemize}
  \item Mark as ``under guardianship''
  \item Verifier requires guardian approval
  \item Small transactions allowed
\end{itemize}

\textbf{Option C}: Defer to Verifier
\begin{itemize}
  \item No Issuer action
  \item Verifier checks guardianship
\end{itemize}

\subsection{Verifier Guardianship Confirmation Flow}

Recommended at contract time:

\begin{enumerate}
  \item Receive ZKP from Holder
  \item Check MerkleRevocationList (standard)
  \item \textbf{New}: Query guardianship status
    \begin{itemize}
      \item Call municipal guardianship confirmation API
      \item Check if DID is under guardianship
    \end{itemize}
\end{enumerate}

Guardianship confirmation API:

\begin{verbatim}
GET /api/v1/guardianship/check/{did}

Response:
{
  "did": "did:amatelus:ward-123...",
  "underGuardianship": true,
  "guardian": {
    "did": "did:amatelus:guardian-789...",
    "type": "Legal Guardian",
    "guardianVC": {
      "issuer": "did:amatelus:city-hall-tokyo",
      "expiresAt": "2026-10-14T23:59:59Z"
    }
  },
  "restrictions": [
    "No new VC issuance",
    "Large contracts require guardian approval",
    "Small transactions (under 10,000 yen) permitted"
  ]
}
\end{verbatim}

\subsection{Contract Cancellation by Guardian}

If ward makes inappropriate contract:

\subsubsection{Legal Basis}

Civil Code Article 9: ``A person under guardianship may cancel their legal acts.''

\subsubsection{Cancellation Procedure}

\begin{enumerate}
  \item Guardian identifies inappropriate contract
  \item Presents Guardian VC to Verifier
  \item Requests contract cancellation and refund
  \item Verifier verifies Guardian VC and executes cancellation
\end{enumerate}

Verifier verification:
\begin{enumerate}
  \item Verify Guardian VC signature
  \item Check \texttt{authority} includes ``contract cancellation''
  \item Verify contract was with ward's DID
  \item Execute cancellation and refund
\end{enumerate}

\section{Security Considerations}

\subsection{Key Leakage Attacks}

\textbf{Attack}: Attacker uses leaked key to generate ZKP

\textbf{Defense}:
\begin{itemize}
  \item Holder immediately revokes old DIDs
  \item Issuer revokes old VCs
  \item Verifier checks current MerkleRevocationList
\end{itemize}

\textbf{Vulnerability window}: Time between leak and revocation completion

\subsection{Fraudulent DID Migration}

\textbf{Attack}: Attacker claims old DID as their own for re-issuance

\textbf{Defense}:
\begin{itemize}
  \item With old key signature: Signature verification prevents fraud
  \item Without old key: Identity credential physical verification prevents fraud
  \item Without identity credential: No defense (information-theoretic impossibility)
\end{itemize}

\subsection{Guardian VC Forgery}

\textbf{Attack}: Attacker forges Guardian VC

\textbf{Defense}:
\begin{itemize}
  \item Guardian VC issued by trusted municipality (cryptographic signature)
  \item Signature forgery requires breaking Dilithium2 (128-bit security)
  \item Recommended: Verify guardianship in legal registry API
\end{itemize}

\section{Implementation Recommendations}

\subsection{Holder Responsibilities}

\begin{itemize}
  \item Store secret key in device secure storage (Keychain, TEE)
  \item Maintain encrypted backups in multiple locations
  \item Obtain identity credential before other VCs
  \item Update DIDs periodically (annually recommended)
  \item Keep Issuer contact list
  \item Test revocation procedures annually
\end{itemize}

\subsection{Issuer Responsibilities}

\begin{itemize}
  \item Define and publish revocation SLA (e.g., 1 hour for high priority)
  \item Implement automatic signature verification
  \item Maintain up-to-date MerkleRevocationLists
  \item Provide 24/7/365 revocation acceptance (high SLA)
  \item Document DID migration trust policies
  \item Keep audit logs of revocation requests and completions
\end{itemize}

\subsection{Verifier Responsibilities}

\begin{itemize}
  \item Always use current MerkleRevocationList
  \item Implement MAX\_VERSION\_LAG (recommended: 5)
  \item For guardianship scenarios: Query guardianship API
  \item Define transaction thresholds for guardian approval
  \item Process contract cancellation requests from guardians
  \item Log all guardian-related transactions
\end{itemize}

\section{Formal Theorems}

\subsection{Theorem: Leakage Recovery Determinism}

With old key signature capability, revocation is cryptographically deterministic.

\begin{lstlisting}[language=Lean]
theorem leakage_revocation_possible
    (holder : Holder)
    (oldSK : SecretKey)
    (oldDID : DID)
    (newDID : DID)
    (timestamp : Nat)
    (h_possession : holder.hasKey oldSK)
    (h_sign : CanSign oldSK (hash (oldDID ++ newDID ++ timestamp))) :
  ∃ (revocationProof : RevocationProof),
    ∀ (issuer : Issuer),
      issuer.hasVC oldDID →
        Verifiable revocationProof ∧
        issuer.canRevoke oldDID revocationProof
\end{lstlisting}

\subsection{Theorem: Recovery Impossibility Without Identity Credential}

\begin{lstlisting}[language=Lean]
theorem recovery_impossible_without_identity_credential
    (holder : Holder)
    (oldDID : DID)
    (newDID : DID)
    (h_no_key : ¬holder.hasKey oldSK_old)
    (h_no_identity_vc : ¬∃ (vc : VC),
      vc.issuer.isIdentityAuthority ∧
      vc.subject = oldDID) :
  ¬∃ (issuer : Issuer),
    issuer.canVerifyRecovery oldDID newDID
\end{lstlisting}

Result: Safe recovery is information-theoretically impossible.

\section{Policy Recommendation Summary}

\subsection{Revocation SLA}

\begin{table}[h]
\centering
\begin{tabular}{|l|l|l|}
\hline
\textbf{Priority} & \textbf{Target} & \textbf{Maximum} \\
\hline
High (leak/theft) & 1 hour & 24 hours \\
Medium (loss) & 24 hours & 7 days \\
Low (planned update) & 7 days & 30 days \\
\hline
\end{tabular}
\end{table}

\subsection{Identity Verification Standards}

\begin{itemize}
  \item Leakage/Update: Signature verification (automatic)
  \item Loss with ID VC: Physical identity document + record linkage
  \item Loss without ID VC: Issuer-specific policy or impossible
  \item Death: Municipal record linkage
  \item Guardianship: Court order + family court registration verification
\end{itemize}

\subsection{Key Design Principles}

\begin{enumerate}
  \item \textbf{Determinism when possible}: Use cryptography to avoid judgment
  \item \textbf{Identity-first}: Obtain identity credential before other VCs
  \item \textbf{Backup preparation}: Require users to backup keys and test recovery
  \item \textbf{Transparency}: Publish all policies and procedures publicly
  \item \textbf{Practical limits}: Accept that some scenarios are unrecoverable
\end{enumerate}
