\chapter{Deployment Models}

\section{Service-Driven Model (Primary)}

In the predominant deployment model:
\begin{enumerate}
  \item Service provider application (bank, government, SNS) initiates
  \item Wallet is invoked from service app via OS-level Intent/deeplink
  \item Wallet generates DIDComm message containing:
    \begin{itemize}
      \item Communication DID (ephemeral or persistent based on service)
      \item DID Document (public key only)
      \item VC/ZKP proving required attributes
    \end{itemize}
  \item DIDComm message sent to service provider (HTTPS, typically)
  \item Service provider performs:
    \begin{itemize}
      \item Cryptographic verification of ZKP/VC
      \item Authorization decision
      \item Service execution
    \end{itemize}
\end{enumerate}

Trust origin: Service provider (centralized, established).

Endpoint lifecycle: Service-scoped (session-based).

\section{Physical Proximity Model (Supplementary)}

For in-person scenarios:
\begin{enumerate}
  \item Bluetooth Low Energy (BLE) provides discovery via physical proximity
  \item Public services (municipal offices, event gates)
  \item Session completes within local network
  \item No persistent endpoint contracts
\end{enumerate}

Examples:
\begin{itemize}
  \item Age verification at municipal counter
  \item Facility access control
  \item In-person credential verification
\end{itemize}

