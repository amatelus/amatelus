\chapter{Trust Architecture}

\section{Responsibility Boundaries}

\begin{center}
\begin{tabular}{|l|c|c|}
\hline
\textbf{Component} & \textbf{AMATELUS} & \textbf{Service Provider} \\
\hline
Public Key Infrastructure & \checkmark & -- \\
DID generation & \checkmark & -- \\
VC issuance/validation (1-layer) & \checkmark & Policy \\
ZKP generation/verification & \checkmark & -- \\
Endpoint management & -- & \checkmark \\
Message delivery & -- & \checkmark \\
Authorization decisions & -- & \checkmark \\
Communication security (TLS) & -- & \checkmark \\
\hline
\end{tabular}
\end{center}

\section{Trust Origin}

\begin{itemize}
  \item \textbf{Cryptographic trust}: Originating from AMATELUS protocol
  \item \textbf{Operational trust}: Originating from service provider (centralized)
  \item \textbf{Authorization trust}: Originating from service provider (centralized)
\end{itemize}

The separation prevents AMATELUS from assuming responsibilities it cannot scale to manage globally.

\section{One-Layer Trust Limitation}

AMATELUS validates only 1-layer VC chains:

\begin{itemize}
  \item \textbf{0-layer}: Direct issuance from trusted anchor
  \item \textbf{1-layer}: Delegated issuance (trustee validated against anchor)
  \item \textbf{2+ layers}: Explicitly \emph{not} validated by AMATELUS protocol
\end{itemize}

This prevents:
\begin{itemize}
  \item Delegation chain attacks
  \item Circular credential verification
  \item Unbounded revocation propagation
\end{itemize}

