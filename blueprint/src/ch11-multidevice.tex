\chapter{Multi-Device Support}


\begin{definition}
  \label{def:multi-chapter}
  This chapter covers multi-device support aspects of AMATELUS.
  \uses{def:amatelus,thm:did-local-generation}
  \lean{AMATELUS.MultiDevice}
  \leanok
\end{definition}
\section{Overview}

The Multi-Device Support specification enables a single Holder to operate wallets with different DIDs on multiple devices and safely transfer claims between devices on a per-claim basis.

\subsection{Background}

Identity-related VCs have the following characteristics:

\begin{itemize}
  \item \textbf{Local Reception Requirement}: Most identity VCs are issued in-person at municipal offices, police stations, etc.
  \item \textbf{Smartphone Reception}: Smartphone is the typical device for in-person VC reception
  \item \textbf{PC Utilization Need}: When participating in the AMATELUS network from home, users typically use a PC
\end{itemize}

Multi-device support is essential to meet these diverse requirements.

\subsection{Design Principles}

\begin{enumerate}
  \item \textbf{Non-Shared Private Keys}: Private keys are never transferred between devices
  \item \textbf{Per-Claim Transfer}: Claims (not entire VCs) are transferred individually
  \item \textbf{Dual Signatures}: Issuer signature + original subject transfer signature
  \item \textbf{W3C Standard Compliance}: Supports \texttt{holder} $\neq$ \texttt{credentialSubject}
  \item \textbf{DIDComm Protocol}: Uses standard DID-to-DID communication protocol
  \item \textbf{End-to-End Encryption}: Transferred data is always encrypted
  \item \textbf{Protocol-Level Rules}: Claims without signatures are ignored
\end{enumerate}

\section{Design Philosophy}

\subsection{Comparison with Trust Chain}

AMATELUS provides two distinct mechanisms:

\begin{table}[h]
\centering
\begin{tabular}{|l|l|l|l|}
\hline
\textbf{Function} & \textbf{Use Case} & \textbf{Mechanism} & \textbf{DID Relationship} \\
\hline
\textbf{Trust Chain} & Schema inheritance, authority delegation & DelegationChain & Different organizational DIDs \\
\hline
\textbf{Multi-Device} & Claim sharing & Per-claim transfer & Same Holder's different DIDs \\
\hline
\end{tabular}
\end{table}

\subsection{Claim Transfer Paths}

In the AMATELUS protocol, claims are transferred via two paths:

\begin{enumerate}
  \item \textbf{Issuer $\rightarrow$ Holder\_A}: Initial claim issuance
  \item \textbf{Holder\_A $\rightarrow$ Holder\_B}: Per-claim transfer (subject of this specification)
\end{enumerate}

External submissions always use only ZKPs; claims themselves are never revealed externally.

\subsection{Claim Structure}

All claims must have an issuer signature:

\begin{verbatim}
structure SignedClaim where
  content : String
  delegationChain : Option DelegationChain
  issuerSignature : Signature
\end{verbatim}

\textbf{Protocol-Level Rule}: Claims without signatures are ignored.

\subsection{Transferred Claim Structure}

Claims transferred between devices carry dual signatures:

\begin{verbatim}
structure TransferredClaim where
  originalClaim : SignedClaim
  originalSubjectDID : ValidDID
  currentHolderDID : ValidDID
  transferProof : Signature
  transferredAt : Nat
\end{verbatim}

\textbf{Dual Signature Roles}:
\begin{enumerate}
  \item \textbf{issuerSignature}: Guarantees claim authenticity (issued by municipality)
  \item \textbf{transferProof}: Proves claim ownership and transfer consent (owned and transferred by original DID)
\end{enumerate}

\subsection{W3C Standard Alignment}

W3C VC standards allow \texttt{holder} (VC possessor) and \texttt{credentialSubject} (VC subject) to be different ($\neq$).

Per-claim transfer maintains:
\begin{itemize}
  \item Issuer signature remains unchanged (original issuer's signature)
  \item \texttt{originalSubjectDID} remains unchanged (original DID maintained)
  \item \texttt{currentHolderDID} is the new device's DID
  \item PC wallet can possess and use for ZKP generation
\end{itemize}

\section{Use Cases}

\subsection{Municipal Certificate Reception}

\begin{enumerate}
  \item Alice brings smartphone to municipal office
  \item Officer verifies smartphone DID: \texttt{did:amt:alice\_smartphone}
  \item Municipality issues residence certificate VC to smartphone wallet
  \item Smartphone wallet receives and stores VC
\end{enumerate}

\subsection{Claim Transfer to Home PC}

\begin{enumerate}
  \item Alice returns home
  \item PC wallet boots: \texttt{did:amt:alice\_pc}
  \item Smartphone wallet initiates claim transfer request to PC wallet
  \item Smartphone wallet generates transfer signature (signed with DID\_A private key)
  \item DIDComm protocol encrypts and transfers claim
  \item PC wallet receives, verifies, and stores claim
  \item PC can now participate in AMATELUS network
\end{enumerate}

\subsection{Claim Utilization (ZKP Generation)}

When PC wallet generates ZKP:

\begin{verbatim}
{
  "public_inputs": {
    "holder_did": "did:amt:alice_pc",
    "age_gte": 20,
    "residence": "Tokyo"
  },
  "proof": "..."
}
\end{verbatim}

\textbf{Important Notes}:
\begin{itemize}
  \item ZKP generation is performed by PC wallet
  \item \texttt{originalSubjectDID} (smartphone's original DID) is secret input
  \item \texttt{currentHolderDID} (PC's DID) is public input
  \item Both issuer and transfer signatures are verified within ZKP
  \item \textbf{Only required claims are input} (other claims are unnecessary)
\end{itemize}

\section{DIDComm Communication Protocol}

\subsection{DIDComm Overview}

\textbf{DIDComm (DID Communication)} is the standard protocol enabling secure communication between DIDs:

\begin{itemize}
  \item \textbf{Specification}: DIDComm Messaging v2.0
  \item \textbf{Features}:
    \begin{itemize}
      \item End-to-end encryption
      \item Authenticated messaging
      \item Transport-agnostic (HTTP, WebSocket, Bluetooth, etc.)
    \end{itemize}
\end{itemize}

\subsection{DIDComm Message Structure}

\subsubsection{Claim Transfer Request}

\begin{verbatim}
{
  "type": "https://amatelus.org/protocols/claim-transfer/2.0/request",
  "id": "uuid-1234-5678",
  "from": "did:amt:alice_pc",
  "to": "did:amt:alice_smartphone",
  "created_time": 1673568000,
  "body": {
    "request_type": "filtered_claims",
    "filters": {
      "content_patterns": ["name", "address"],
      "issuer": "did:amt:municipality123"
    }
  }
}
\end{verbatim}

\subsubsection{Claim Transfer Response}

\begin{verbatim}
{
  "type": "https://amatelus.org/protocols/claim-transfer/2.0/response",
  "from": "did:amt:alice_smartphone",
  "to": "did:amt:alice_pc",
  "body": {
    "transferred_claims": [
      {
        "original_claim": {
          "content": "{\"name\": \"Alice\", \"address\": \"Tokyo\"}",
          "issuer_signature": "..."
        },
        "original_subject_did": "did:amt:alice_smartphone",
        "current_holder_did": "did:amt:alice_pc",
        "transfer_proof": "...",
        "transferred_at": 1673568010
      }
    ],
    "success": true
  }
}
\end{verbatim}

\subsection{Authentication Flow}

\subsubsection{Device Pairing}

Initial pairing of two wallets:

\begin{enumerate}
  \item PC displays QR code (DID + temporary token)
  \item Smartphone scans QR code
  \item Smartphone sends DIDComm connection request
  \item PC prompts user for approval
  \item User approves on PC
  \item Both devices add peer DID to trusted list
\end{enumerate}

\subsubsection{Mutual Authentication}

Communication between paired devices:

\begin{enumerate}
  \item Request side signs DIDComm message
  \item Response side verifies signature and checks trusted list
  \item Response side sends encrypted response
  \item Request side decrypts and verifies response
\end{enumerate}

\section{Claim Transfer Mechanism}

\subsection{Transferred Data}

Data sent during claim transfer:

\begin{verbatim}
{
  "original_claim": {
    "content": "{\"name\": \"Alice\", \"address\": \"Tokyo\"}",
    "issuer_signature": "..."
  },
  "original_subject_did": "did:amt:alice_smartphone",
  "current_holder_did": "did:amt:alice_pc",
  "transfer_proof": "...",
  "transferred_at": 1673568010
}
\end{verbatim}

\textbf{Unchanged Elements}:
\begin{itemize}
  \item \texttt{original\_claim.content}: Claim content
  \item \texttt{original\_claim.issuer\_signature}: Issuer signature
  \item \texttt{original\_subject\_did}: Original holder DID (smartphone)
\end{itemize}

\textbf{Added Elements}:
\begin{itemize}
  \item \texttt{transfer\_proof}: Transfer signature by original subject (required)
  \item \texttt{current\_holder\_did}: Current holder DID (PC)
  \item \texttt{transferred\_at}: Transfer timestamp
\end{itemize}

\subsection{Transfer Signature Generation}

Smartphone wallet generates transfer signature:

\begin{verbatim}
def prepareClaimTransfer
    (claim : SignedClaim)
    (originalSubjectDID : ValidDID)
    (currentHolderDID : ValidDID)
    (timestamp : Nat) : TransferredClaim :=
  let message := encodeTransferMessage claim.content
                   originalSubjectDID currentHolderDID
  let transferProof := sign message originalSubjectSecretKey
  {
    originalClaim := claim,
    originalSubjectDID := originalSubjectDID,
    currentHolderDID := currentHolderDID,
    transferProof := transferProof,
    transferredAt := timestamp
  }
\end{verbatim}

\subsection{Receiving Wallet Validation}

PC wallet validates received claim:

\begin{verbatim}
def validateClaim
    (tc : TransferredClaim)
    (issuerDID : ValidDID)
    (trustedAnchors : List ValidDID) : Bool :=
  let issuerSigValid := tc.originalClaim.verify issuerDID
  let transferSigValid := tc.verifyTransferProof
  let issuerTrusted := trustedAnchors.contains issuerDID
  issuerSigValid && transferSigValid && issuerTrusted
\end{verbatim}

\subsection{Receiving Wallet Storage}

PC wallet stores received claim in this structure:

\begin{verbatim}
{
  "claim_id": "uuid-claim-001",
  "original_claim": {
    "content": "{\"name\": \"Alice\"}",
    "issuer_signature": "..."
  },
  "original_subject_did": "did:amt:alice_smartphone",
  "current_holder_did": "did:amt:alice_pc",
  "transfer_proof": "...",
  "storage_metadata": {
    "received_at": "2025-01-13T12:00:00Z",
    "received_from": "did:amt:alice_smartphone",
    "issuer_did": "did:amt:municipality123"
  }
}
\end{verbatim}

\section{ZKP Efficiency}

\subsection{Per-Claim Transfer Advantages}

\textbf{Problem with V1.0 (VC-based transfer)}:
\begin{itemize}
  \item Transferring entire VC requires all claims as ZKP inputs during generation
  \item Unnecessary claims bloat the circuit size
  \item High computational cost with privacy concerns
\end{itemize}

\textbf{Advantages of V2.0 (Per-claim transfer)}:
\begin{itemize}
  \item Only required claims input to ZKP circuit
  \item Circuit size minimized, computational cost reduced
  \item Enhanced privacy (unnecessary claims not handled)
  \item Dual signatures provide complete security guarantee
\end{itemize}

\subsection{ZKP Generation Process}

When PC wallet generates ZKP:

\begin{verbatim}
structure ZKPSecretInputsForTransferredClaim where
  claimContent : String
  issuerSignature : Signature
  transferSignature : Signature
  originalSubjectDID : ValidDID

structure ZKPPublicInputsForTransferredClaim where
  currentHolderDID : ValidDID
  publicAttributes : List (String $\times$ String)
\end{verbatim}

\textbf{ZKP Circuit Verification}:
\begin{enumerate}
  \item \texttt{issuerSignature} is valid (municipality issued)
  \item \texttt{transferSignature} is valid (DID\_A owns and transferred)
  \item \texttt{originalSubjectDID} matches subject in claim (integrity)
\end{enumerate}

\subsection{Computational Cost Comparison}

\begin{table}[h]
\centering
\begin{tabular}{|l|l|l|l|}
\hline
\textbf{Method} & \textbf{Input Claims} & \textbf{Circuit Size} & \textbf{Computation Cost} \\
\hline
\textbf{V1.0 (VC transfer)} & All claims (5) & Large & High \\
\hline
\textbf{V2.0 (Per-claim transfer)} & Required only (1) & Small & Low \\
\hline
\end{tabular}
\end{table}

\subsection{Privacy Enhancement}

Per-claim transfer enables:
\begin{itemize}
  \item Excluding unnecessary claims from ZKP circuit
  \item Hiding existence of unnecessary claims
  \item Improved selective disclosure granularity
\end{itemize}

\section{Security Considerations}

\subsection{Private Key Non-Sharing}

\textbf{Critical Principle}: Private keys are never transferred between devices

\begin{itemize}
  \item Each device maintains independent DID and key pair
  \item Only claim data (issuer signature + transfer signature) is transferred
  \item Private key compromise impact is limited to that device
\end{itemize}

\subsection{Dual Signature Security}

Per-claim transfer provides complete security through dual signatures:

\begin{enumerate}
  \item \textbf{issuerSignature}:
    \begin{itemize}
      \item Guarantees claim authenticity
      \item Proves issuance by issuer (municipality)
      \item Prevents tampering
    \end{itemize}

  \item \textbf{transferProof}:
    \begin{itemize}
      \item Proves claim ownership
      \item Proves original subject (DID\_A) owned it
      \item Proves transfer consent
    \end{itemize}
\end{enumerate}

\textbf{Security Theorem}:
\begin{verbatim}
theorem claim_transfer_preserves_issuer_signature :
  forall (claim : SignedClaim)
    (originalSubjectDID currentHolderDID : ValidDID)
    (timestamp : Nat),
  let tc := prepareClaimTransfer claim originalSubjectDID
               currentHolderDID timestamp
  tc.originalClaim.issuerSignature = claim.issuerSignature
\end{verbatim}

\subsection{End-to-End Encryption}

All claim transfers use DIDComm encryption:

\begin{enumerate}
  \item Sender encrypts with recipient's public key
  \item Transport layer applies additional encryption (optional, e.g., TLS)
  \item Recipient decrypts with own private key
\end{enumerate}

\subsection{Device Authentication}

\subsubsection{Pairing-Time Authentication}

\begin{itemize}
  \item QR code + temporary token
  \item Explicit user approval
  \item Registration in trusted list
\end{itemize}

\subsubsection{Transfer-Time Authentication}

\begin{itemize}
  \item Trusted list verification
  \item DIDComm signature verification
  \item Timestamp verification (replay attack prevention)
\end{itemize}

\subsection{Claim Integrity Verification}

Receiving wallet verifies:

\begin{verbatim}
def validateClaim
    (tc : TransferredClaim)
    (issuerDID : ValidDID)
    (trustedAnchors : List ValidDID) : Bool :=
  let issuerSigValid := tc.originalClaim.verify issuerDID
  let transferSigValid := tc.verifyTransferProof
  let issuerTrusted := trustedAnchors.contains issuerDID
  issuerSigValid && transferSigValid && issuerTrusted
\end{verbatim}

\subsection{Man-in-the-Middle (MITM) Attack Mitigation}

DIDComm protocol provides:

\begin{enumerate}
  \item \textbf{AEAD}: Tampering detection
  \item \textbf{DID Signatures}: Sender authenticity guarantee
  \item \textbf{Pairing Confirmation}: User approval
  \item \textbf{Dual Signatures}: Complete integrity guarantee
\end{enumerate}

\subsection{Privacy Protection}

\begin{itemize}
  \item \textbf{Claim Transfer Secrecy}: Transfer fact unknown to third parties
  \item \textbf{originalSubjectDID Protection}: Secret input during ZKP generation
  \item \textbf{Selective Transfer}: Only required claims transferred
  \item \textbf{Metadata Separation}: Transfer history stored locally only
\end{itemize}

\subsection{Protocol-Level Guarantee}

\textbf{Claims without signatures are ignored}:
\begin{itemize}
  \item All claims must have issuer signature
  \item Transferred claims must have transfer signature
  \item Unsigned claims automatically rejected at protocol level
  \item Tampering attempts automatically fail
\end{itemize}

\section{Implementation Guidelines}

\subsection{Wallet Implementation Requirements}

\subsubsection{Essential Features}

\begin{enumerate}
  \item \textbf{DIDComm Support}
    \begin{itemize}
      \item Implement DIDComm v2.0 protocol
      \item Support end-to-end encryption
    \end{itemize}

  \item \textbf{Device Management}
    \begin{itemize}
      \item Manage trusted device list
      \item Implement pairing (QR code, etc.)
    \end{itemize}

  \item \textbf{Claim Transfer}
    \begin{itemize}
      \item Support claim sending (with filtering)
      \item Generate transfer signatures (original subject's private key)
      \item Support claim receiving (with dual signature verification)
    \end{itemize}

  \item \textbf{ZKP Generation Extension}
    \begin{itemize}
      \item Support \texttt{currentHolderDID} as public input
      \item Support \texttt{originalSubjectDID} as secret input
      \item Verify both issuer and transfer signatures in ZKP
      \item Input only required claims (others unnecessary)
    \end{itemize}

  \item \textbf{Signature Verification}
    \begin{itemize}
      \item Verify issuer signatures
      \item Verify transfer signatures
      \item Reject unsigned claims (protocol-level)
    \end{itemize}
\end{enumerate}

\subsubsection{Recommended Features}

\begin{enumerate}
  \item \textbf{Selective Transfer}
    \begin{itemize}
      \item Transfer specific claims only
      \item Content pattern filtering
      \item Issuer-based filtering
    \end{itemize}

  \item \textbf{Transfer History}
    \begin{itemize}
      \item Log device and transfer timestamp
      \item Support transfer revocation (revocation notification)
    \end{itemize}

  \item \textbf{Automatic Synchronization}
    \begin{itemize}
      \item Auto-transfer new claims
      \item Configure inter-device sync
    \end{itemize}

  \item \textbf{Claim Management}
    \begin{itemize}
      \item Per-claim storage and search
      \item Display delegation chains
      \item Claim selection UI for ZKP generation
    \end{itemize}
\end{enumerate}

\subsection{Transport Layer Options}

DIDComm supports multiple transports:

\begin{table}[h]
\centering
\begin{tabular}{|l|l|l|}
\hline
\textbf{Transport} & \textbf{Use Case} & \textbf{Characteristics} \\
\hline
\textbf{HTTP/HTTPS} & Internet via cloud & Cloud relay possible \\
\hline
\textbf{WebSocket} & Real-time communication & Bidirectional \\
\hline
\textbf{Bluetooth} & Local communication & No internet required \\
\hline
\textbf{NFC} & Proximity communication & Very short range \\
\hline
\end{tabular}
\end{table}

Recommended configuration:
\begin{itemize}
  \item Same network: WebSocket (fast)
  \item Different networks: HTTPS (stable)
  \item Offline: Bluetooth (no internet)
\end{itemize}

\subsection{Error Handling}

\subsubsection{Transfer Failure Response}

\begin{verbatim}
{
  "type": "https://amatelus.org/protocols/claim-transfer/2.0/error",
  "thid": "uuid-1234-5678",
  "body": {
    "error_code": "SIGNATURE_VERIFICATION_FAILED",
    "error_message": "Signature verification failed",
    "details": {
      "claim_id": "uuid-claim-001",
      "issuer": "did:amt:municipality123"
    }
  }
}
\end{verbatim}

\subsubsection{Retry Logic}

\begin{enumerate}
  \item Detect transfer failure
  \item Retry with exponential backoff (max 3 times)
  \item Notify user on final failure
  \item Log failure
\end{enumerate}

\subsection{Testing Strategy}

\subsubsection{Functional Testing}

\begin{itemize}
  \item Pairing success/failure
  \item Claim transfer success/failure
  \item Issuer signature verification
  \item Transfer signature verification
  \item Dual signature verification
  \item Filtering functionality
  \item Unsigned claim rejection
\end{itemize}

\subsubsection{Security Testing}

\begin{itemize}
  \item MITM attack simulation
  \item Invalid claim rejection
  \item Signature tampering detection
  \item Untrusted device rejection
  \item Automatic unsigned claim rejection
\end{itemize}

\subsubsection{Performance Testing}

\begin{itemize}
  \item Large volume claim transfer performance
  \item Encryption/decryption overhead
  \item Network latency handling
  \item ZKP generation time comparison (V1.0 vs V2.0)
\end{itemize}

\subsubsection{ZKP Efficiency Testing}

\begin{itemize}
  \item Circuit size comparison by claim count
  \item Performance with required claims only
  \item Comparison with all claims
\end{itemize}

\section{Future Extensions}

\subsection{Cloud Synchronization}

Use trusted cloud service as relay:

\begin{verbatim}
Smartphone -> [Encrypted] -> Cloud -> [Decrypted] -> PC
\end{verbatim}

Requirements:
\begin{itemize}
  \item End-to-end encryption mandatory
  \item Cloud cannot access claim content
  \item Zero-knowledge proof-based backup
  \item Per-claim synchronization
\end{itemize}

\subsection{Group Wallet}

Safe claim sharing within family or organization:

\begin{verbatim}
Parent Wallet <-> Child Wallet (guardian function)
Corporate Wallet <-> Employee Wallet (role proof)
\end{verbatim}

Requirements:
\begin{itemize}
  \item Per-claim sharing control
  \item Dual signatures guarantee integrity
  \item Improved selective disclosure granularity
\end{itemize}

\subsection{Conditional Transfer}

Enable claim transfer only under specific conditions:

\begin{verbatim}
{
  "transfer_policy": {
    "allowed_devices": ["did:amt:alice_pc", "did:amt:alice_tablet"],
    "allowed_time": "09:00-18:00",
    "require_biometric": true,
    "max_transfers": 3,
    "allowed_claims": ["name", "address"]
  }
}
\end{verbatim}

\subsection{Trust Chain Integration}

Combine N-level delegation with per-claim transfer:

\begin{itemize}
  \item Transfer claims containing delegation chains
  \item Verify delegation chains at transfer destination
  \item ZKP efficiency (required claims only)
\end{itemize}

\section{Formal Verification}

Theorems formally proven in \texttt{AMATELUS/MultiDevice.lean}:

\begin{enumerate}
  \item \textbf{claim\_transfer\_preserves\_issuer\_signature}
    \begin{itemize}
      \item Issuer signature preserved during transfer
    \end{itemize}

  \item \textbf{claim\_transfer\_preserves\_content}
    \begin{itemize}
      \item Claim content preserved during transfer
    \end{itemize}

  \item \textbf{transferred\_claim\_has\_transfer\_proof}
    \begin{itemize}
      \item Transferred claims always have transfer signature
    \end{itemize}

  \item \textbf{device\_trust\_symmetric}
    \begin{itemize}
      \item Device trust verification symmetry
    \end{itemize}

  \item \textbf{valid\_claim\_stays\_valid\_after\_transfer}
    \begin{itemize}
      \item Valid claims remain valid after transfer
    \end{itemize}
\end{enumerate}

These theorems formally guarantee the security of per-claim transfer.
