\chapter{Introduction}

\section{AMATELUS Protocol Overview}

\begin{definition}
  \label{def:amatelus}
  The AMATELUS protocol is a cryptographic authentication mechanism integrating:
  \begin{itemize}
    \item Decentralized Identifiers (DIDs)
    \item Verifiable Credentials (VCs)
    \item Zero-Knowledge Proofs (ZKPs)
  \end{itemize}
  \lean{AMATELUS.Core}
  \leanok
\end{definition}

\subsection{What AMATELUS Provides}

\begin{proposition}
  \label{prop:amatelus-capabilities}
  AMATELUS delivers the following cryptographic capabilities:
  \begin{itemize}
    \item Public key infrastructure (PKI) based distributed identifiers
    \item Zero-knowledge proof-based attribute ownership verification
    \item Verifiable credentials for claim issuance and storage
    \item Cryptographic trust guarantees
  \end{itemize}
  \uses{def:amatelus}
  \lean{AMATELUS.Capabilities}
  \leanok
\end{proposition}

\subsection{What AMATELUS Does NOT Provide}

\begin{proposition}
  \label{prop:amatelus-non-scope}
  AMATELUS \emph{does not} provide:
  \begin{itemize}
    \item Centralized directory services (DID resolution is local-only)
    \item Mediation or relay infrastructure
    \item User authorization decisions
    \item Communication endpoint management
    \item Message delivery guarantees
  \end{itemize}
  \uses{def:amatelus}
  \lean{AMATELUS.NonScope}
\end{proposition}

\subsection{Design Philosophy}

\begin{proposition}
  \label{prop:design-philosophy}
  AMATELUS is a \emph{cryptographic authentication mechanism}, not a communication infrastructure.
  \uses{def:amatelus, prop:amatelus-capabilities, prop:amatelus-non-scope}
  \lean{AMATELUS.Philosophy}
  \leanok
\end{proposition}

\begin{theorem}
  \label{thm:boundary-clarity}
  The boundary between cryptographic verification (AMATELUS responsibility) and
  service provision (service provider responsibility) eliminates distributed
  infrastructure complexity while preserving cryptographic security properties.
  \uses{prop:design-philosophy}
  \proves{prop:design-philosophy}
  \lean{AMATELUS.BoundaryClarity}
  \leanok
\end{theorem}

