\chapter{Cryptographic Foundations}

\section{Security Assumptions}

\begin{definition}
  \label{def:crypto-assumptions}
  AMATELUS relies on the following cryptographic security assumptions:
  \begin{itemize}
    \item \textbf{Collision-resistant hash}: SHA3-512 provides 128-bit security against quantum adversaries
    \item \textbf{Unforgeable signatures}: Dilithium2 provides 128-bit security against quantum adversaries
    \item \textbf{ZKP soundness}: Standard zero-knowledge properties (completeness, soundness, zero-knowledge)
    \item \textbf{Independent nonce generation}: Cryptographically random nonces with sufficient entropy
  \end{itemize}
  \uses{def:amatelus}
  \lean{AMATELUS.CryptoAssumptions}
  \leanok
\end{definition}

\section{Threat Model and Mitigations}

\subsection{Impersonation Attack (Different Secret Key)}

\begin{theorem}
  \label{thm:impersonation-resistance}
  Impersonation attacks with different secret keys are cryptographically prevented.

  If an attacker uses a different secret key to forge a ZKP, the signature verification will fail
  because DIDComm makes the sender's public key known to the recipient.
  \uses{def:crypto-assumptions}
  \proves{prop:amatelus-capabilities}
  \lean{AMATELUS.ImpersonationResistance}
  \leanok
\end{theorem}

\subsection{Replay Attack (Same ZKP, Same User)}

\begin{theorem}
  \label{thm:replay-resistance}
  Replay attacks are prevented by dual nonce binding.

  Each ZKP corresponds to a unique nonce pair. Even if a legitimate user's ZKP is
  reused in a new session, the nonces will differ, preventing the attack.
  \uses{def:crypto-assumptions}
  \proves{prop:amatelus-capabilities}
  \lean{AMATELUS.ReplayResistance}
  \leanok
\end{theorem}

\subsection{Man-in-the-Middle Attack}

\begin{proposition}
  \label{prop:mitm-defense}
  Man-in-the-Middle attacks are mitigated at the transport layer.

  While AMATELUS provides cryptographic identity verification, ECDH-1PU authenticated
  encryption and TLS/HTTPS are the responsibility of service providers.
  \uses{def:crypto-assumptions}
  \lean{AMATELUS.MITMDefense}
\end{proposition}

\subsection{Sybil Attack (Multiple DIDs)}

\begin{proposition}
  \label{prop:sybil-resilience}
  Multiple DID possession is intentional protocol design for privacy protection.

  While a single entity can control multiple DIDs, Anonymous Hash Identifiers (AHI)
  restrict per-audit-domain abuse through cryptographic binding to national identity systems.
  \uses{def:amatelus}
  \lean{AMATELUS.SybilResilience}
\end{proposition}

