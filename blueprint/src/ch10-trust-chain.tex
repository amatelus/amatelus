\chapter{Trust Chain Architecture}

\section{Overview}

The Trust Chain specification provides a hierarchical delegation mechanism that enables authority transfer through cryptographically verified credentials. Key capabilities include:

\begin{itemize}
  \item \textbf{Authority Delegation}: Upper organizations delegate authority to lower organizations
  \item \textbf{Cryptographic Verification}: Delegation chains are cryptographically verifiable
  \item \textbf{Schema-Based Delegation}: Delegation content is structurally defined using JSON Schema
  \item \textbf{Dynamic Depth Limitation}: Delegators specify \texttt{maxDepth}; Nat monotonic decrease prevents infinite hierarchies
  \item \textbf{Holder-Centric Design}: VCs can be issued by anyone (including Holders themselves); verifiers check only the delegation chain's \texttt{grantorDID}
  \item \textbf{Per-Claim Signatures}: Recipients sign each claim individually, dramatically reducing VP/ZKP input sizes
  \item \textbf{ZKP-Only Submission}: VPs are internal structures; all submissions use only ZKPs
  \item \textbf{Field-Level Selective Disclosure}: ZKPs enable revelation of only required fields, maximizing privacy
\end{itemize}

\section{Trust Chain Components}

\subsection{Architecture}

A trust chain flows from a root authority through delegatees to holders:

\begin{center}
\begin{verbatim}
Root Anchor (Trusted Authority)
    v Delegation Credential
    | (delegated authority -> recipient)
    v
Delegatee (Authority)
    v Attribute Credential
    | (claim with embedded delegation)
    v
Holder (Subject)
    v Re-packaging (optional)
    | (Holder re-issues with own signature)
    v
Verifier (Service Provider)
    verifies:
    - ZKP validity
    - grantorDID in trustedAnchors
    does NOT verify:
    - VC issuer
    - granteeDID matching
\end{verbatim}
\end{center}

\subsection{Critical Design Principles}

\begin{itemize}
  \item \textbf{VC Issuer Flexibility}: The \texttt{issuer} field can be anyone; trust derives from the delegation chain's \texttt{grantorDID}
  \item \textbf{Delegation Chain Focus}: Verifiers inspect \texttt{grantorDID} (trust root), not \texttt{issuer}
  \item \textbf{ZKP Compatibility}: VCs issued by Holders can still prove delegated authority via ZKP
  \item \textbf{Holder Re-packaging}: Holders can re-issue received VCs with their own signature; each claim's proof remains unchanged
\end{itemize}

\section{Delegation Credentials}

\subsection{Structure}

A Delegation Credential contains multiple delegations, each authorizing a specific claim type:

\begin{itemize}
  \item \textbf{issuer}: Delegator's DID (must be in \texttt{trustedAnchors})
  \item \textbf{credentialSubject.id}: Original recipient's DID (historical information)
  \item \textbf{credentialSubject.delegations}: Array of delegation objects
  \item \textbf{proof}: Delegator's signature over the entire credential
\end{itemize}

Each delegation element contains:

\begin{itemize}
  \item \textbf{grantorDID}: Authority granting the right (verification anchor)
  \item \textbf{granteeDID}: Initial recipient DID (historical information)
  \item \textbf{label}: Human-readable claim label (e.g., ``Resident Certificate'')
  \item \textbf{claimSchema}: JSON Schema (AMATELUS Subset) defining allowed claim structure
  \item \textbf{maxDepth}: Maximum further delegation depth (Nat $\geq 1$)
  \item \textbf{proof}: Delegator's signature over the delegation object
\end{itemize}

\subsection{Dynamic Depth Limitation}

The protocol prevents infinite delegation chains through monotonic depth decrease:

\begin{enumerate}
  \item Each delegator specifies \texttt{maxDepth} (maximum delegations from this point forward)
  \item Each recipient computes: \texttt{nextDepth = min(parentDepth - 1, delegation.maxDepth)}
  \item When depth reaches zero, further delegation is impossible
  \item Lean 4 formally proves termination via \texttt{termination\_by} clause
\end{enumerate}

Example: Government $\xrightarrow{\text{maxDepth}=5}$ Prefecture $\xrightarrow{\text{maxDepth}=2}$ Municipality $\xrightarrow{\text{maxDepth}=1}$ Department $\xrightarrow{\text{maxDepth}=0}$ (chain stops)

\section{Attribute Credentials with Embedded Delegation}

\subsection{Direct Issuance (0-Layer)}

Standard W3C VC issued directly by a trusted authority:

\begin{itemize}
  \item No delegation chain
  \item Simple structure: \texttt{issuer}, \texttt{credentialSubject.claims}, \texttt{proof}
  \item Verification: \texttt{issuer} must be in \texttt{trustedAnchors}
\end{itemize}

\subsection{Delegated Issuance (1-Layer and Beyond)}

When issued under delegated authority, each claim embeds delegation information:

\begin{itemize}
  \item \textbf{content}: Actual claim data
  \item \textbf{delegation}: Reference to the delegation authorization
  \item \textbf{delegationProof}: Grantor's signature over the delegation
  \item \textbf{proof}: Grantee's signature over the content (prevents Holder modification)
\end{itemize}

Verification checks:
\begin{enumerate}
  \item \texttt{delegation.grantorDID} is in \texttt{trustedAnchors}
  \item \texttt{content} conforms to \texttt{delegation.claimSchema}
  \item \texttt{delegationProof} is valid under \texttt{grantorDID}
  \item \texttt{proof} is valid under \texttt{granteeDID} (enables ZKP verification)
\end{enumerate}

Does \textbf{NOT} verify:
\begin{itemize}
  \item VC's \texttt{issuer} field (can be anyone, including the Holder)
  \item \texttt{granteeDID} matches VC's \texttt{issuer}
\end{itemize}

\subsection{Holder Re-Packaging}

Holders can re-issue received VCs under their own DID:

\begin{itemize}
  \item VC's \texttt{issuer} becomes Holder's DID
  \item VC's \texttt{proof} becomes Holder's signature
  \item Each claim's \texttt{content}, \texttt{delegation}, \texttt{delegationProof}, \texttt{proof} remain unchanged
  \item Verification remains unchanged (still trusts \texttt{grantorDID})
\end{itemize}

This enables Holders to maintain privacy from service providers while proving delegated authority through ZKP.

\section{Verifiable Presentations and ZKP Generation}

\subsection{VP as Internal Structure}

Unlike W3C specifications, AMATELUS uses VPs internally only:

\begin{itemize}
  \item VPs are \textbf{never submitted directly} to issuers or verifiers
  \item VPs serve to organize claims before ZKP generation
  \item Holders extract only required claims from multiple VCs and assemble them into a VP
  \item ZKP is generated from the VP's selected claims, not the original VCs
\end{itemize}

\subsection{ZKP Input Size Reduction}

This design dramatically reduces computational cost:

\begin{itemize}
  \item \textbf{Traditional method}: ZKP includes all claims from all VCs
  \item \textbf{VP+Individual Signatures method}: ZKP includes only selected claims
  \item \textbf{Efficiency gain}: If Holder needs 3 claims from 23 total, ZKP input shrinks to $\approx 1/8$
  \item \textbf{Mobile impact}: Battery consumption, memory use, and response time all improve proportionally
\end{itemize}

Each claim's individual \texttt{proof} (recipient's signature) enables this efficiency:

\begin{itemize}
  \item Allows claims to be verified independently
  \item Prevents Holder from modifying claim content (signature would fail)
  \item Enables selective disclosure without revealing unneeded claims
\end{itemize}

\subsection{Submission to Issuers}

When requesting a new VC from an issuer:

\begin{enumerate}
  \item Holder creates a VP (internal only) from relevant credentials
  \item Holder generates a ZKP with public inputs including:
    \begin{itemize}
      \item Holder's DID (issuer needs to record who is requesting)
      \item Required fields (e.g., name, age category, income threshold proof)
      \item \texttt{grantorDIDs} (proof authorities)
    \end{itemize}
  \item Holder submits: \texttt{\{zkp, publicInputs\}} only
  \item Issuer verifies ZKP and creates new VC with \texttt{credentialSubject.id} from public input
\end{enumerate}

\subsection{Submission to Verifiers}

When proving attributes to a service provider:

\begin{enumerate}
  \item Holder creates a VP (internal only) from relevant credentials
  \item Holder generates a ZKP with public inputs including:
    \begin{itemize}
      \item Required properties only (e.g., \texttt{age $\geq$ 20}, \texttt{residence in Japan})
      \item Holder's DID is \textbf{hidden} in ZKP's secret inputs
      \item \texttt{grantorDIDs} (proof authorities)
    \end{itemize}
  \item Holder submits: \texttt{\{zkp, publicInputs\}} only
  \item Verifier verifies ZKP without learning:
    \begin{itemize}
      \item Holder's identity (DID)
      \item Specific personal information (exact birthdate, address, etc.)
      \item Which claims or VCs were used
      \item Original credential content
    \end{itemize}
\end{enumerate}

This achieves complete privacy for cross-service authentication: the verifier learns only what is cryptographically necessary.

\section{Security Properties}

\subsection{Depth Limitation Proofs}

Lean 4 formally proves three key properties:

\begin{enumerate}
  \item \textbf{Termination}: \texttt{verifyChain} always completes in finite time (via \texttt{termination\_by remainingDepth})
  \item \textbf{Finite Chain Length}: No valid chain can exceed \texttt{initialMaxDepth}
  \item \textbf{N-Layer Finiteness}: Multi-layer delegations remain bounded
\end{enumerate}

\subsection{Circular Delegation Prevention}

DIDs in a chain must be unique (O(n) check):

\begin{itemize}
  \item Detects cycles by comparing \texttt{getAllDIDs(chain)} with \texttt{getAllDIDs(chain).eraseDups}
  \item \texttt{verifyChain} returns false if duplicates found
\end{itemize}

\subsection{Holder-Centric Verification}

By focusing verification on \texttt{grantorDID} rather than \texttt{issuer}:

\begin{itemize}
  \item \textbf{Issuer Independence}: Credentials can be re-issued by different parties without verification failure
  \item \textbf{Holder Privacy}: Holders can re-package credentials under their own DID without disclosure
  \item \textbf{ZKP Compatibility}: Delegation authority persists even when issuer is hidden via ZKP
\end{itemize}

\subsection{Per-Claim Signature Guarantees}

Each claim's \texttt{proof} (recipient's signature):

\begin{itemize}
  \item Prevents Holders from modifying claim content
  \item Enables independent claim verification (critical for VP-based ZKP generation)
  \item Survives Holder re-packaging (signature remains unchanged)
\end{itemize}

\section{Examples}

\subsection{Single-Layer Delegation}

Government delegates ``Resident Certificate'' authority to municipalities:

\begin{center}
\begin{verbatim}
Government (grantorDID)
  v Delegation Credential
    (grants Resident Certificate authority)
    maxDepth = 1
  v
Municipality (granteeDID)
  v Attribute Credential
    (issues resident data with embedded delegation)
  v
Resident (Holder)
  v Requests Bank Account
    (verifier: Bank)
  v
Bank
  - Verifies ZKP
  - Confirms government DID in trustedAnchors
  - Approves account
\end{verbatim}
\end{center}

\subsection{Multi-Layer Delegation with Holder Privacy}

\begin{center}
\begin{verbatim}
Government (grantorDID, maxDepth=3)
  v Prefecture (maxDepth=2)
  v Municipality (maxDepth=1)
  v Resident (Holder, issues self-signed VC)
  v Applies for Job
    (submits ZKP: age >= 18, no criminal record)
  v
Employer (Verifier)
  - Verifies ZKP
  - Does NOT learn:
    - Resident's identity
    - Exact age/birthdate
    - Criminal history status
    - Original credentials
  - Learns only:
    - Age threshold met
    - Background check passed
    - Government issued proof
\end{verbatim}
\end{center}

