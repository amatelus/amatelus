\chapter{Privacy Architecture}


\begin{definition}
  \label{def:priv-chapter}
  This chapter covers privacy architecture aspects of AMATELUS.
  \uses{def:amatelus,thm:did-local-generation}
  \lean{AMATELUS.Privacy}
  \leanok
\end{definition}
\section{Anti-Linkability Across Services}

Multiple DIDs enable cross-service unlinkability:

\begin{verbatim}
For all DID_1, DID_2, Service_1, Service_2:
  (Service_1 != Service_2) AND
  Link(DID_1, DID_2) requires 2^128 quantum operations
\end{verbatim}

\section{Zero-Knowledge Property}

ZKP reveal attribute ownership without revealing identity:

\begin{itemize}
  \item \textbf{Public}: Attribute claimed (age $\geq$ 18)
  \item \textbf{Hidden}: Identity proving the attribute
  \item \textbf{Hidden}: Secret key generating the proof
  \item \textbf{Verified}: ZKP authenticity via public key
\end{itemize}

\subsection{Privacy Preservation Under DIDComm}

A critical design question: How can DIDComm's requirement for explicit public key transmission
(for impersonation prevention) coexist with privacy protection?

\begin{theorem}
  \label{thm:privacy-with-didcomm}
  Privacy is maintained despite public key transmission via DIDComm due to:

  \begin{itemize}
    \item \textbf{Ephemeral Communication DIDs}: Each session uses distinct DIDs with different key pairs
    \item \textbf{Different services, different keys}: User generates new keys for each service
    \item \textbf{Cryptographic unlinkability}: Public keys from different sessions cannot be linked
      (requires $2^{128}$ quantum operations to link, assuming quantum-resistant hash functions)
    \item \textbf{Service-specific communication DIDs}: Verifier knows only the communication DID,
      not the user's persistent identity DID
  \end{itemize}

  \textbf{Result}: While DIDComm requires public key disclosure for impersonation prevention,
  the use of ephemeral DIDs with distinct key pairs per service ensures cross-service
  unlinkability and privacy.

  \lean{AMATELUS.Privacy}
  \leanok
\end{theorem}

\subsection{Application-Layer Privacy Considerations}

Applications using AMATELUS should consider additional privacy measures:

\begin{itemize}
  \item \textbf{Session management}: Create new communication DIDs for each session to prevent cross-session linkability
  \item \textbf{Retention policies}: Do not retain communication DIDs or ZKPs longer than necessary
  \item \textbf{Logging}: Minimize logging of ZKPs and DIDDocuments; log only what is necessary for application function
  \item \textbf{Nonce handling}: If implementing nonce mechanisms (application responsibility), handle nonce values with privacy in mind
\end{itemize}

\section{Deniable Authentication}

For privacy-sensitive scenarios, anonymous encryption (Anoncrypt) available:

\begin{itemize}
  \item Sender identity hidden from intermediaries
  \item Recipient verifies proof authenticity (still authenticated)
  \item Sender maintains plausible deniability
\end{itemize}

